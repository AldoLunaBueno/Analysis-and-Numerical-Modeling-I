%! Aldo Luna, Alexandra Mazzetti, Carlos Aznarán, Edward Canales
%! Universidad Nacional de Ingeniería
%! Facultad de Ciencias
%! Lima, Perú
%! Uso:
%! $ sudo pacman -Syu texlive-most zathura # instalar las dependencias y un visor
%! $ arara tercera
%! $ zathura tercera.pdf
%! Ver https://wiki.archlinux.org/title/TeX_Live
% arara: clean: {
% arara: --> extensions:
% arara: --> ['aux','log','nav',
% arara: --> 'out','snm','toc','pdf']
% arara: --> }
% arara: lualatex: {
% arara: --> interaction: batchmode
% arara: --> }
% arara: lualatex: {
% arara: --> interaction: batchmode
% arara: --> }
% arara: clean: {
% arara: --> extensions:
% arara: --> ['aux','log','nav',
% arara: --> 'out','snm','toc']
% arara: --> }
\PassOptionsToPackage{svgnames}{xcolor}
\documentclass[
	spanish,
	8pt,
	utf8,
	xcolor=table,
	handout,
	aspectratio=169,
	professionalfonts,
	notheorems,
	mathserif,
	% t
]{beamer}
\setbeamersize{text margin left=0pt,text margin right=0pt}
\usepackage[spanish,es-sloppy]{babel}
\spanishdatedel\decimalpoint
\usepackage{mathtools}
\usepackage{nicematrix}
\usepackage{systeme}
\usepackage{enumerate}
\usepackage{multicol}
\usepackage{array}
\usepackage[linesnumbered,ruled,boxed,vlined,spanish]{algorithm2e}
\usepackage{algorithmicx}

\newcolumntype{x}[1]{>{\centering\arraybackslash\hspace{0pt}}p{#1}}

\newcounter{savedenum}
\newcommand*{\saveenum}{\setcounter{savedenum}{\theenumi}}
\newcommand*{\resume}{\setcounter{enumi}{\thesavedenum}}

\setbeamertemplate{navigation symbols}{}
\setbeamertemplate{footline}{}
\setbeamertemplate{headline}{}

\begin{document}

\begin{frame}
	\begin{enumerate}
		\item

		      Usando la eliminación gaussiana, factorice las siguientes
		      matrices en la forma $A=LU$, donde $L$ es una matriz
		      triangular inferior unitaria y $U$ es una matriz triangular
		      superior.

		      \begin{multicols}{2}
			      \begin{enumerate}[a)]

				      \item

				            \begin{math}
					            \begin{bNiceMatrix}
						            3 & 0  & 3 \\
						            0 & -1 & 3
						            \\ 1 & 3 & 0
					            \end{bNiceMatrix}
				            \end{math}.

				      \item

				            \begin{math}
					            \begin{bNiceMatrix}
						            1 & 0  & \frac{1}{3} & 0  \\
						            0 & 1  & 3           & -1 \\
						            3 & -3 & 0           & 6  \\
						            0 & 2  & 4           & -6
					            \end{bNiceMatrix}
				            \end{math}.
			      \end{enumerate}
		      \end{multicols}
		\item

		      Considere la matriz
		      \begin{math}
			      \begin{bNiceMatrix}
				      2 & 2 & 1 \\
				      1 & 1 & 1 \\
				      3 & 2 & 1
			      \end{bNiceMatrix}
		      \end{math}.

		      \begin{enumerate}[a)]

			      \item

			            Demuestre que $A$ no se puede factorizar en el
			            producto de una matriz triangular inferior
			            unitaria y una matriz triangular superior.

			      \item

			            Intercambie las filas de $A$ para que esto se pueda
			            hacer.
		      \end{enumerate}

		\item

		      Use el algoritmo de factorización $LDL^{t}$ y obtenga una
		      factorización de la matriz $A$:

		      \begin{multicols}{2}
			      \begin{enumerate}[a)]
				      \item

				            \begin{math}
					            \begin{bNiceMatrix}
						            4 & 1  & 1  & 1 \\
						            1 & 3  & -1 & 1 \\
						            1 & -1 & 2  & 0 \\
						            1 & 1  & 0  & 2
					            \end{bNiceMatrix}
				            \end{math}.

				      \item

				            \begin{math}
					            \begin{bNiceMatrix}
						            6  & 2 & 1  & -1 \\
						            2  & 4 & 1  & 0  \\
						            1  & 1 & 4  & -1 \\
						            -1 & 0 & -1 & 3
					            \end{bNiceMatrix}
				            \end{math}.
			      \end{enumerate}
		      \end{multicols}

		\item

		      Use el algoritmo de Choleski y obtenga una factorización de
		      la forma $A=LL^{t}$ para la matrices.


		      \begin{multicols}{2}
			      \begin{enumerate}[a)]
				      \item

				            \begin{math}
					            \begin{bNiceMatrix}
						            4 & 1  & 1  & 1 \\
						            1 & 3  & -1 & 1 \\
						            1 & -1 & 2  & 0 \\
						            1 & 1  & 0  & 2
					            \end{bNiceMatrix}
				            \end{math}.

				      \item

				            \begin{math}
					            \begin{bNiceMatrix}
						            6  & 2 & 1  & -1 \\
						            2  & 4 & 1  & 0  \\
						            1  & 1 & 4  & -1 \\
						            -1 & 0 & -1 & 3
					            \end{bNiceMatrix}
				            \end{math}.
			      \end{enumerate}
		      \end{multicols}
		      \saveenum
	\end{enumerate}
\end{frame}

\begin{frame}
	\begin{enumerate}
		\resume

		\item
		      Use la factorización de Crout con sistemas tridiagonales
		      para resolver los siguientes sistemas lineales:

		      \begin{multicols}{2}
			      \begin{enumerate}[a)]
				      \item

				            \begin{math}
					            \systeme{
					            x_{1}-
					            x_{2}=
					            0,
					            -2x_{1}+
					            4x_{2}-
					            2x_{3}=
					            1,
					            -x_{2}+
					            2x_{3}=
					            1.5
					            }
				            \end{math}.

				      \item

				            \begin{math}
					            \systeme{
						            0.5x_{1}+
						            0.25x_{2}=
						            0.35,
						            0.35x_{1}+
						            0.8x_{2}+
						            0.4x_{3}=
						            0.77,
						            0.25x_{2}+
						            x_{3}+
						            0.5x_{4}=
						            -0.5,
						            x_{3}-
						            2x_{4}=
						            -2.25
					            }
				            \end{math}.
			      \end{enumerate}
		      \end{multicols}

		\item

		      Implemente el algoritmo de la descomposición de Schur y
		      aplíquelo a las siguientes matrices

		      \begin{multicols}{2}
			      \begin{enumerate}[a)]
				      \item

				            \begin{math}
					            A=
					            \begin{bNiceMatrix}
						            5  & 7  \\
						            -2 & -4
					            \end{bNiceMatrix}
				            \end{math}.

				      \item


				            \begin{math}
					            B=
					            \begin{bNiceMatrix}
						            0.9 & 0 \\
						            -1  & 2
					            \end{bNiceMatrix}
				            \end{math}.
			      \end{enumerate}
		      \end{multicols}

		\item

		      Una compañía minera trabaja en 3 minas, cada una de las
		      cuales produce minerles de tres clases.
		      La primera mina puede producir 4 toneladas del mineral A, 3
		      toneladas del mineral B, y 5 toneladas del mineral C; la
		      segunda mina puede producir 1 tonelada de cada uno de los
		      minerales y la tercera mina, 2 toneladas del A,
		      4 toneladas del B y 3 toneladas del C, por cada hora de
		      funcionamiento.
		      Se desea satisfacer los tres pedidos siguientes
		      \begin{table}[ht!]
			      \centering
			      \tiny
			      \begin{tabular}{|c|c|c|c|}
				      \hline
				      Pedidos & Mineral A & Mineral B & Mineral C \\
				      \hline
				      $P_{1}$ & $19$      & $25$      & $25$      \\
				      \hline
				      $P_{2}$ & $13$      & $16$      & $16$      \\
				      \hline
				      $P_{3}$ & $8$       & $12$      & $10$      \\
				      \hline
			      \end{tabular}
		      \end{table}

		      Determine
		      \begin{multicols}{2}

			      \begin{enumerate}[a)]
				      \item

				            Modele el sistema a resolver.

				      \item

				            Resolver usando los programas $LDL^{T}$ y Cholesky.
			      \end{enumerate}
		      \end{multicols}

		\item

		      Determine el número de diagonales de un polígono convexo de $n$ lados.

		      \begin{multicols}{2}

			      \begin{enumerate}[a)]
				      \item

				            Modele el sistema a resolver.

				      \item

				            Resolver usando los programas $LDL^{T}$ y Cholesky.
			      \end{enumerate}
		      \end{multicols}
		      \saveenum
	\end{enumerate}
\end{frame}

\begin{frame}
	\begin{enumerate}
		\resume

		\item

		      Una familia consta de una madre, un padre y una hija.
		      La suma de las edades actuales de los 3 es de 80 años.
		      Dentro de 22 años, la edad del hijo será la mitad que la de la madre.
		      Si el padre es un año mayor que la madre.
		      Determinar la edad de la familia según lo siguientes requerimientos

		      \begin{multicols}{2}

			      \begin{enumerate}[a)]
				      \item

				            Modele el sistema a resolver.

				      \item

				            Resolver usando los programas $LDL^{T}$ y Cholesky.
			      \end{enumerate}
		      \end{multicols}

		\item

		      En la empresa plástica Elsa se fabrican tres tipos de
		      productos: botellas, garrafas y bidones.
		      Se utiliza como materia prima 10 kg de granza de
		      polietileno cada hora.
		      Se sabe que para fabricar cada botella se necesitan 50
		      gramos de granza, para cada garrafa 100 gramos y para cada
		      bidón 1 kg.
		      El gerente también nos dice que se debe producir el doble
		      de botellas que de garrafas.
		      Por último, se sabe que por motivos de capacidad de trabajo
		      en las máquinas se producen en total 52 productos cada hora.
		      Se desea conocer la producción en cada hora según el
		      siguiente requerimiento

		      \begin{multicols}{2}

			      \begin{enumerate}[a)]
				      \item

				            Modele el sistema a resolver.

				      \item

				            Resolver usando los programas $LDL^{T}$ y Cholesky.
			      \end{enumerate}
		      \end{multicols}

		\item

		      En una heladería, por un helado, dos zumos y 4 batidos nos cobraron 35 soles.
		      Otro día, por 4 helados, 4 zumos y un batido nos cobraron 34 soles.
		      Un tercer día por 2 helados, 3 zumos y 4 batidos 42 soles.
		      Determine el precio de cada uno según el siguiente requerimiento

		      \begin{multicols}{2}

			      \begin{enumerate}[a)]
				      \item

				            Modele el sistema a resolver.

				      \item

				            Resolver usando los programas $LDL^{T}$ y Cholesky.
			      \end{enumerate}
		      \end{multicols}

		      \saveenum
	\end{enumerate}
\end{frame}

\begin{frame}
	\begin{enumerate}
		\resume

		\item

		      Dado la matriz
		      \begin{math}
			      A=
			      \begin{bNiceMatrix}
				      1 & 4  & 2  & 3  \\
				      4 & 27 & 16 & 13 \\
				      2 & 16 & 10 & 7  \\
				      3 & 13 & 7  & 7
			      \end{bNiceMatrix}
		      \end{math}

		      determine la descomposición Parlett-Reid.

		\item

		      Dado la matriz
		      \begin{math}
			      A=
			      \begin{bNiceMatrix}
				      0 & 1 & 2 & 3 \\
				      1 & 2 & 2 & 2 \\
				      2 & 2 & 3 & 3 \\
				      3 & 2 & 3 & 4
			      \end{bNiceMatrix}
		      \end{math}

		      determine la descomposición Parlett-Reid.

		\item

		      Dado la matriz
		      \begin{math}
			      A=
			      \begin{bNiceMatrix}
				      1  & 10 & 20 \\
				      10 & 1  & 30 \\
				      20 & 30 & 1
			      \end{bNiceMatrix}
		      \end{math}
		      determine la descomposición Parlett-Reid.


		\item

		      Dado la matriz

		      \begin{math}
			      A=
			      \begin{bNiceMatrix}
				      0,0909 & 0,1111  & 0,1429  & 0,2000  & 0,3333  & 1,0000  \\
				      0,1111 & 0,1429  & 0,2000  & 0,3333  & 1,0000  & -1,0000 \\
				      0,1429 & 0,2000  & 0,3333  & 1,0000  & -1,0000 & -0,3333 \\
				      0,2000 & 0,3333  & 1,0000  & -1,0000 & -0,3333 & -0,2000 \\
				      0,3333 & 1,0000  & -1,0000 & -0,3333 & -0,2000 & -0,1429 \\
				      1,0000 & -1,0000 & -0,3333 & -0,2000 & -0,1429 & -0,1111
			      \end{bNiceMatrix}
		      \end{math}

		      determine la descomposición Parlett-Reid.

		\item

		      Dado la matriz
		      \begin{math}
			      A=
			      \begin{bNiceMatrix}
				      1 & 1 & 1 \\
				      1 & 1 & 1 \\
				      1 & 1 & 1
			      \end{bNiceMatrix}
		      \end{math}
		      determine la descomposición Parlett-Reid.

		\item


		      Encuentre la descomposición SVD de la matriz

		      \begin{math}
			      A=
			      \begin{bNiceMatrix}
				      0 & 1 & 0 & 0 \\
				      0 & 0 & 2 & 0 \\
				      0 & 0 & 0 & 3 \\
				      0 & 0 & 0 & 0
			      \end{bNiceMatrix}
		      \end{math}.

		      \saveenum
	\end{enumerate}
\end{frame}
% minerals game sprites
\end{document}