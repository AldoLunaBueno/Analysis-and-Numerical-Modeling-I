%! Aldo Luna, Alexandra Mazzetti, Carlos Aznarán, Edward Canales
%! Universidad Nacional de Ingeniería
%! Facultad de Ciencias
%! Lima, Perú
%! Uso:
%! $ sudo pacman -Syu texlive-most zathura # dependencias, visor
%! $ arara solution_tercera
%! $ zathura solution_tercera.pdf
%! Ver https://wiki.archlinux.org/title/TeX_Live
% arara: clean: {
% arara: --> extensions:
% arara: --> ['aux','log','nav',
% arara: --> 'out','snm','toc','pytxcode','pdf']
% arara: --> }
% arara: lualatex: {
% arara: --> shell: yes,
% arara: --> draft: yes,
% arara: --> interaction: batchmode
% arara: --> }
%! arara: biber
%! arara: pythontex
% arara: lualatex: {
% arara: --> shell: yes,
% arara: --> draft: yes,
% arara: --> interaction: batchmode
% arara: --> }
% arara: lualatex: {
% arara: --> shell: yes,
% arara: --> synctex: yes,
% arara: --> interaction: batchmode
% arara: --> }
% arara: clean: {
% arara: --> extensions:
% arara: --> ['aux','log','nav',
% arara: --> 'out','snm','toc','pytxcode']
% arara: --> }
\PassOptionsToPackage{svgnames}{xcolor}
\documentclass[
	spanish,
	8pt,
	utf8,
	xcolor=table,
	handout,
	aspectratio=169,
	professionalfonts,
	% notheorems,
	mathserif,
	leqno,
	% t
]{beamer}
\setbeamersize{text margin left=5pt,text margin right=5pt}
\usepackage[spanish,es-sloppy]{babel}
\spanishdatedel\decimalpoint
\usepackage{mathtools}
\usepackage{nicematrix}
\usepackage{systeme}
\usepackage{minted}
\usepackage{enumerate}
\usepackage{multicol}
% \usepackage{pythontex}
\usepackage[
	citestyle=numeric,
	style=apa,
	backend=biber,
	defernumbers=true,
	sorting=ynt,
	maxcitenames=4
]{biblatex}
\addbibresource{references.bib}

\newcolumntype{x}[1]{>{\centering\arraybackslash\hspace{0pt}}p{#1}}

\newcounter{savedenum}
\newcommand*{\saveenum}{\setcounter{savedenum}{\theenumi}}
\newcommand*{\resume}{\setcounter{enumi}{\thesavedenum}}

\setbeamertemplate{navigation symbols}{}
\setbeamertemplate{footline}{}
\setbeamertemplate{headline}{}

% https://tex.stackexchange.com/questions/68080/beamer-bibliography-icon
\setbeamertemplate{bibliography item}{%
	\ifboolexpr{ test {\ifentrytype{book}} or test {\ifentrytype{mvbook}}
		or test {\ifentrytype{collection}} or test {\ifentrytype{mvcollection}}
		or test {\ifentrytype{reference}} or test {\ifentrytype{mvreference}} }
	{\setbeamertemplate{bibliography item}[book]}
	{\ifentrytype{online}
		{\setbeamertemplate{bibliography item}[online]}
		{\setbeamertemplate{bibliography item}[article]}}%
	\usebeamertemplate{bibliography item}}

\defbibenvironment{bibliography}
{\list{}
	{\settowidth{\labelwidth}{\usebeamertemplate{bibliography item}}%
		\setlength{\leftmargin}{\labelwidth}%
		\setlength{\labelsep}{\biblabelsep}%
		\addtolength{\leftmargin}{\labelsep}%
		\setlength{\itemsep}{\bibitemsep}%
		\setlength{\parsep}{\bibparsep}}}
{\endlist}
{\item}

\makeatletter
\newenvironment<>{proofs}[1][\proofname]{%
    \par
    \def\insertproofname{#1\@addpunct{.}}%
    \usebeamertemplate{proof begin}#2}
  {\usebeamertemplate{proof end}}
\makeatother


\title{
	\huge\sffamily
	Práctica Dirigida
}

\subtitle{
	\large\scshape
	Análisis y Modelamiento Numérico I\quad CM4F1 B\\[.5\baselineskip]
		\normalsize\normalfont
		Profesor Ángel Enrique Ramírez Gutiérrez.
}

\author{
	Carlos A. Aznarán Laos
}

\institute{\large
	Facultad de Ciencias \and
	Universidad Nacional de Ingeniería
}
\date{\today}

\begin{document}

\frame{\titlepage}
\begin{frame}

	\begin{enumerate}\setcounter{enumi}{0}
		\item

		      Muestre que el número de condición $\kappa$ para el producto
		      $Ax$ (con respecto a la perturbación $x$) es
		      \begin{equation*}
			      \kappa\left(x\right)=
			      \left\|A\right\|
			      \frac{
				      \left\|x\right\|
			      }{
				      \left\|Ax\right\|
			      }.
		      \end{equation*}

	\end{enumerate}

	\begin{solution}
		Sean $A\in\mathbb{R}^{m\times n}$ y $x\in\mathbb{R}^{n\times 1}$.
		El \alert{número de condición relativo} $\kappa\left(x\right)$
		del producto $Ax$ es
		\begin{align*}
			\kappa\left(x\right)
			 & \coloneqq
			\lim\limits_{\delta\to0}\sup\limits_{h<\delta}
			\left\|
			\frac{\frac{A\left(x+h\right)-A\left(x\right)}{Ax}}{\frac{h}{x}}
			\right\|.                                                                                                                \\
			\kappa\left(x\right)
			 & =
			\sup_{h\neq0}
			\frac{
				\frac{\left\|A\left(x+h\right)-Ax\right\|}{\alert{\left\|Ax\right\|}}
			}{
				\frac{\left\|h\right\|}{\alert{\left\|x\right\|}}
			}.
			\shortintertext{Como $\left\|Ax\right\|$ y $\left\|x\right\|$ no dependen de $h$, podemos factorizar}
			\kappa\left(x\right)
			 & =
			\frac{\alert{\left\|x\right\|}}{\alert{\left\|Ax\right\|}}\sup_{h\neq0}\frac{\left\|Ax+Ah-Ax\right\|}{\left\|h\right\|}. \\
			\kappa\left(x\right)
			 & =
			\alert{\sup_{h\neq0}\frac{\left\|Ah\right\|}{\left\|h\right\|}}\frac{\left\|x\right\|}{\left\|Ax\right\|}.               \\
			\shortintertext{Un factor es la \alert{norma matricial de $A$}}
			\kappa\left(x\right)
			 & =
			\alert{\left\|A\right\|}\frac{\left\|x\right\|}{\left\|Ax\right\|}.
		\end{align*}
	\end{solution}
\end{frame}

\end{document}


2. Let $U=\left(u_{i j}\right)$ be a nonsingular upper triangular matrix. Then show that:
$$
	K_{\infty}(U) \geq \frac{\max \left(\left|u_{i i}\right|\right)}{\min \left(\left|u_{i i}\right|\right)} .
$$
3. Resuelva los siguientes sistemas. Use eliminación Gaussiana con pivoteo parcial mediante la factorización $P A=L U$.
a) $\left.\left(\begin{array}{ccc}-1 & 1 & -4 \\ 2 & 2 & 0 \\ 3 & 3 & 2\end{array}\right)\left(\begin{array}{l}x_1 \\ x_2 \\ x_3\end{array}\right)=\left(\begin{array}{c}0 \\ 1 \\ \frac{1}{2}\end{array}\right) . \quad c\right)\left(\begin{array}{lll}1 & 6 & 0 \\ 2 & 1 & 0 \\ 0 & 2 & 1\end{array}\right)\left(\begin{array}{l}x_1 \\ x_2 \\ x_3\end{array}\right)=\left(\begin{array}{l}3 \\ 1 \\ 1\end{array}\right)$
4. Find a singular value decomposition for $A=\left(\begin{array}{ccc}1 & 0 & 1 \\ -1 & 1 & 0\end{array}\right)$.
5. Find a singular value decomposition for $A=\left(\begin{array}{ccc}3 & 2 & 2 \\ 2 & 3 & -2\end{array}\right)$.
6. Prove that if $A$ has this property (unit row diagonally dominant):
$$
	a_{i i}=1>\sum_{j=1, j \neq i}^n\left|a_{i j}\right| \quad(1 \leq i \leq n)
$$
then the Richardson method is successful.
7. Sea $\|\cdot\|$ una norma matricial inducida ysea $S$ una matriz no singular. Defina:
$$
	\|A\|^{\prime}=\left\|S A S^{-1}\right\| .
$$
Demuestre que $\|A\|^{\prime}$ es una norma matricial inducida.
8. Sea $A$ una matriz definida positiva $\mathrm{y}$ defina la función:

$$
	q(x)=\langle x, A x\rangle-2\langle x, b\rangle
$$
Demuestre que en el método de máximo descenso se cumple:
$$
	q\left(x^{(k+1)}\right)=q\left(x^{(k)}\right)-\frac{\left\|r^{(k)}\right\|^4}{\left\langle r^{(k)}, A r^{(k)}\right.},
$$
donde $r^{(k)}=b-A x^{(k)}$.
9. Considere el sistema lineal:
$$
	\left(\begin{array}{ccc}
			2  & 0   & -1 \\
			-2 & -10 & 0  \\
			-1 & -1  & 4
		\end{array}\right)\left(\begin{array}{l}
			x_1 \\
			x_2 \\
			x_3
		\end{array}\right)=\left(\begin{array}{c}
			1   \\
			-12 \\
			2
		\end{array}\right)
$$
Realice dos iteraciones del método de Jacobi, Gauss-Seidel y Gradiente Conjugado iniciando en $x^{(0)}=$ $(0,0,0)^T$.
10. Let $A$ be symmetric, positive definite and of order $n$. Let $\left\{v_1, \ldots, v_n\right\}$ be an $A$-orthogonal set in $\mathbb{R}^n$, with all $v_i \neq 0$. Define:
$$
	Q_j=\frac{v_j v_j^T A}{v_j^T A v_j} \quad j=1, \ldots, n
$$
Prove the following properties for $Q_j$ :
a) $Q_j v_i=0$ if $i \neq j$ and $Q_j v_j=v_j$.
b) $Q_j^2=Q_j$.
c) $\left\langle x, Q_j y\right\rangle_A=\left\langle Q_j x, y\right\rangle_A$, forall $x, y \in \mathbb{R}^n$.
d) $\left\langle Q_j x,\left(I-Q_j\right) y\right\rangle_A=0$, for all $x, y \in \mathbb{R}^n$.
e) $\|x\|_A^2=\left\|Q_j x\right\|_A^2+\left\|\left(I-Q_j\right) x\right\|_A^2$, for all $x \in \mathbb{R}^n$.

11. El sistema lineal:
$$
	\begin{aligned}
		x_1+\frac{1}{2} x_2             & =\frac{5}{21}  \\
		\frac{1}{2} x_1+\frac{1}{3} x_2 & =\frac{11}{84}
	\end{aligned}
$$
tiene solución $\left(x_1, x_2\right)=(1 / 6,1 / 7)^T$.
a) Resuelva el sistema lineal usando eliminación Gaussiana con dos dígitos y redondeo.
b) Resuelva el sistema lineal usando el método del gradiente conjugado con dos dígitos.
c) ¿Cuál método da una mejor respuesta?