%! Aldo Luna, Alexandra Mazzetti, Carlos Aznarán, Edward Canales
%! Universidad Nacional de Ingeniería
%! Facultad de Ciencias
%! Lima, Perú
%! Uso:
%! $ sudo pacman -Syu texlive-most zathura # instalar las dependencias y un visor
%! $ arara segunda
%! $ zathura segunda.pdf
%! Ver https://wiki.archlinux.org/title/TeX_Live
% arara: clean: {
% arara: --> extensions:
% arara: --> ['aux','log','nav',
% arara: --> 'out','snm','toc','pdf']
% arara: --> }
% arara: lualatex: {
% arara: --> interaction: batchmode
% arara: --> }
% arara: lualatex: {
% arara: --> interaction: batchmode
% arara: --> }
% arara: clean: {
% arara: --> extensions:
% arara: --> ['aux','log','nav',
% arara: --> 'out','snm','toc']
% arara: --> }
\PassOptionsToPackage{svgnames}{xcolor}
\documentclass[
	spanish,
	8pt,
	utf8,
	xcolor=table,
	handout,
	aspectratio=169,
	professionalfonts,
	notheorems,
	mathserif,
	% t
]{beamer}
\setbeamersize{text margin left=0pt,text margin right=0pt}
\usepackage[spanish,es-sloppy]{babel}
\spanishdatedel\decimalpoint
\usepackage{mathtools}
\usepackage{nicematrix}
\usepackage{systeme}
\usepackage{enumerate}
\usepackage{multicol}
\usepackage{array}
\usepackage[linesnumbered,ruled,boxed,vlined,spanish]{algorithm2e}
\usepackage{algorithmicx}

\newcolumntype{x}[1]{>{\centering\arraybackslash\hspace{0pt}}p{#1}}

\newcounter{savedenum}
\newcommand*{\saveenum}{\setcounter{savedenum}{\theenumi}}
\newcommand*{\resume}{\setcounter{enumi}{\thesavedenum}}

\setbeamertemplate{navigation symbols}{}
\setbeamertemplate{footline}{}
\setbeamertemplate{headline}{}

\begin{document}

\begin{frame}
	\begin{enumerate}
		\item

		      Obtenga los números de condición relativos de las
		      siguientes funciones

		      \begin{multicols}{4}
			      \begin{enumerate}[a)]

				      \item

				            \begin{math}
					            f\left(x\right)=
					            x^{p}.
				            \end{math}

				      \item

				            \begin{math}
					            f\left(x\right)=
					            \log\left(x\right).
				            \end{math}


				      \item

				            \begin{math}
					            f\left(x\right)=
					            \cos\left(x\right).
				            \end{math}

				      \item

				            \begin{math}
					            f\left(x\right)=
					            \exp\left(x\right).
				            \end{math}
			      \end{enumerate}
		      \end{multicols}
		\item

		      Suponga que $f$ y $g$ son funciones de valor real que
		      tienen números de condición $\kappa_{f}$ y $\kappa_{g}$,
		      respectivamente.
		      Defina una nueva función
		      \begin{math}
			      h\left(x\right)=
			      \left(f\circ g\right)
			      \left(x\right)
		      \end{math}.
		      Muestre que para $x$ en el dominio de $h$, el número de
		      condición de $h$ satisface
		      \begin{equation*}
			      \kappa_{h}\left(x\right)=
			      \kappa_{f}\left(g\left(x\right)\right)\cdot
			      \kappa_{g}\left(x\right).
		      \end{equation*}

		\item

		      Usando la regla de la cadena, obtenga los números de
		      condición relativos de las siguientes funciones
		      \begin{multicols}{3}
			      \begin{enumerate}[a)]

				      \item

				            \begin{math}
					            f\left(x\right)=
					            \sqrt{x+5}.
				            \end{math}

				      \item

				            \begin{math}
					            f\left(x\right)=
					            \exp\left(-x^{2}\right).
				            \end{math}


				      \item

				            \begin{math}
					            f\left(x\right)=
					            \cos\left(2\pi x\right).
				            \end{math}
			      \end{enumerate}
		      \end{multicols}

		\item

		      Suponga que $f$ es una función con número de condición
		      $\kappa_{f}$, y que $f^{-1}$ es su función inversa.
		      Muestre que el número de condición de $f^{-1}$ satisface

		      \begin{equation*}
			      \kappa_{f^{-1}}\left(x\right)=
			      \frac{1}{\kappa_{f}\left(f^{-1}\left(x\right)\right)},
		      \end{equation*}

		      siempre que el denominador sea distinto de cero.


		\item

		      El polinomio $x^{2}-2x+1$ tiene una raíz doble $r=1$.

		      \begin{enumerate}[a)]

			      \item\label{poly:i}

			      Usando una computadora o calculadora, haga una
			      tabla de las raíces de
			      $x^{2}-\left(2+\epsilon\right)x+1$ para
			      $\epsilon=10^{-4},10^{-6},\ldots,10^{-12}$.

			      \

			      \item

			            ¿Qué parecen implicar los resultados del
			            inciso~\eqref{poly:i} acerca del número de
			            condición de la raíz?
		      \end{enumerate}
		      \saveenum
	\end{enumerate}
\end{frame}

\begin{frame}
	\begin{enumerate}
		\resume

		\item
		      Determine el Landau de las siguientes funciones

		      \begin{multicols}{4}
			      \begin{enumerate}[a)]

				      \item

				            \begin{math}
					            \dfrac{1}{n^{2}}.
				            \end{math}

				      \item

				            \begin{math}
					            \cos\left(n\right).
				            \end{math}


				      \item

				            \begin{math}
					            \sin\left(\dfrac{x}{n}\right).
				            \end{math}

				      \item

				            \begin{math}
					            \sqrt{n+1}-\sqrt{n}.
				            \end{math}
			      \end{enumerate}
		      \end{multicols}

		\item

		      La pérdida de cifras significativas se puede evitar
		      reordenando los cálculos.
		      Determine en los siguientes casos una forma equivalente que
		      evite la pérdida de cifras significativas para valores
		      indicados de $x$.

		      \begin{multicols}{4}
			      \begin{enumerate}[a)]

				      \item

				            \begin{math}
					            \ln\left(x+1\right)-\ln\left(x\right).
				            \end{math}

				      \item

				            \begin{math}
					            \sqrt{x^{2}+1}-x.
				            \end{math}


				      \item

				            \begin{math}
					            1-\cos\left(x\right).
				            \end{math}

				      \item

				            \begin{math}
					            \sin\left(x\right)-x.
				            \end{math}
			      \end{enumerate}
		      \end{multicols}

		\item

		      Sea $f\colon\mathbb{R}^{n}\to\mathbb{R}$ definida por

		      \begin{multicols}{3}
			      \begin{enumerate}[a)]

				      \item

				            \begin{math}
					            f\left(x\right)=
					            \dfrac{x}{4}
				            \end{math},
				            con $n=1$.

				      \item

				            \begin{math}
					            f\left(x\right)=
					            \sqrt{x}
				            \end{math},
				            con $n=1$.


				      \item

				            \begin{math}
					            f\left(x_{1},x_{2}\right)=x_{1}\cdot x_{2}
				            \end{math},
				            con $n=2$.
			      \end{enumerate}
		      \end{multicols}

		      Determine el número de condición.

		\item

		      En un aparcamiento hay 55 vehículos entre coches y motos.
		      Si el total de ruedas es de 170.
		      Determine el número de coches y motos que hay según el
		      siguiente requerimiento

		      \begin{multicols}{2}
			      \begin{enumerate}[a)]

				      \item

				            Modele el problema.

				      \item

				            Determine la norma matricial de $A$.


				      \item

				            Determine el número de condicionamiento de $A$.

				      \item

				            Indique si está bien o mal condicionado.
			      \end{enumerate}
		      \end{multicols}

		\item

		      Un fabricante de bombillas gana 0,3 dólares por cada
		      bombilla que sale de la fábrica, pero pierde 0,4 dólares
		      por cada una que sale defectuosa.
		      Un día en el que fabricó 2100 bombillas obtuvo un beneficio
		      de 484,4 dólares.
		      Determine el número de bombillas buenas y defectuosa según
		      el siguiente requerimiento

		      \begin{multicols}{2}
			      \begin{enumerate}[a)]

				      \item

				            Modele el problema.

				      \item

				            Determine la norma matricial de $A$.


				      \item

				            Determine el número de condicionamiento de $A$.

				      \item

				            Indique si está bien o mal condicionado.
			      \end{enumerate}
		      \end{multicols}
		      \saveenum
	\end{enumerate}
\end{frame}

\begin{frame}
	\begin{enumerate}
		\resume

		\item

		      Suponga que $A=uv^{t}$ donde $u$ y $v$ son vectores.
		      Demuestre que
		      \begin{math}
			      {\left\|A\right\|}_{2}=
			      \left\|u\right\|
			      \left\|v\right\|
		      \end{math},
		      donde
		      \begin{math}
			      \left\|A\right\|_{2}\coloneqq
			      \max\limits_{x\neq0}
			      \dfrac{\left\|Ax\right\|}{\left\|x\right\|}
		      \end{math}
		      (norma espectral)
		      y $\left\|\cdot\right\|$ representa la norma euclidiana.


		\item

		      Dados
		      \begin{equation*}
			      A=
			      \frac{1}{2}
			      \begin{bNiceMatrix}
				      1   & 1   \\
				      1+a & 1-a
			      \end{bNiceMatrix},\quad
			      A^{-1}=
			      \frac{1}{a}
			      \begin{bNiceMatrix}
				      a-1 & 1  \\
				      a+1 & -1
			      \end{bNiceMatrix},
		      \end{equation*}

		      donde $a$ es pequeño y distinto de cero
		      (por ejemplo $a=10^{-10}$).
		      Demuestre que el número de condición
		      $\kappa\left(A\right)\geq\dfrac{1}{\left|a\right|}$,
		      donde
		      \begin{math}
			      \kappa\left(A\right)=
			      {\left\|A\right\|}_{2}
			      {\left\|A^{-1}\right\|}_2
		      \end{math}.

		\item

		      Suponga que $A=UBV$ con $U$, $V$ ortogonales y $B$ no singular.
		      Demuestre que $\kappa\left(A\right)=\kappa\left(B\right)$.

		\item

		      Determine el número de condición de la matriz dada a continuación
		      \begin{equation*}
			      A=
			      \begin{bNiceMatrix}
				      1      & 2 \\
				      1,0001 & 2
			      \end{bNiceMatrix}.
		      \end{equation*}

		\item

		      Utilice la eliminación gaussiana con sustitución hacia
		      atrás y aritmética de redondeo de dos dígitos para
		      resolver los siguientes sistemas lineales.
		      No reordene las ecuaciones.
		      (La solución exacta de cada sistema es $x_{1}=-1$, $x_{2}=1$, $x_{3}=3$.)

		      \begin{multicols}{2}
			      \begin{enumerate}[a)]
				      \item

				            \begin{math}
					            \systeme{
					            -x_{1}+
					            4x_{2}+
					            x_{3}=
					            8,
					            \frac{5}{3}x_{1}+
					            \frac{2}{3}x_{2}+
					            \frac{2}{3}x_{3}=
					            1,
					            2x_{1}+
					            x_{2}+
					            4x_{3}=
					            11
					            }
				            \end{math}.

				      \item


				            \begin{math}
					            \systeme{
					            4x_{1}+
					            2x_{2}-
					            x_{3}=
					            -5,
					            \frac{1}{9}x_{1}+
					            \frac{1}{9}x_{2}-
					            \frac{1}{3}x_{3}=
					            -1,
					            x_{1}+
					            4x_{2}+
					            2x_{3}=
					            9
					            }
				            \end{math}.
			      \end{enumerate}
		      \end{multicols}
		      \saveenum
	\end{enumerate}
\end{frame}

\begin{frame}
	\begin{enumerate}
		\resume

		\item

		      Sean dos números tales que la suma de un tercio del primero
		      más un quinto del segundo sea igual a 13 y que si se
		      multiplica el primero por 5 y el segundo por 7 se obtiene
		      247 como suma de los dos productos.
		      Determine los números según el requerimiento siguiente.

		      \begin{multicols}{2}
			      \begin{enumerate}[a)]
				      \item

				            Modele el problema.

				      \item

				            Determine la norma matricial de $A$ y $A^{-1}$.


				      \item

				            Determine la solución usando los métodos de Gauss
				            y Gauss-Jordan.
			      \end{enumerate}
		      \end{multicols}


		\item

		      El perímetro de un rectángulo es 64 cm y la diferencia entre
		      las medidas de la base y la altura es 6 cm.
		      Determine las dimensiones de dicho rectángulo según el
		      requerimiento siguiente.

		      \begin{multicols}{2}
			      \begin{enumerate}[a)]
				      \item

				            Modele el problema.

				      \item

				            Determine la norma matricial de $A$ y $A^{-1}$.

				      \item

				            Determine la solución usando los métodos de Gauss y Gauss-Jordan.
			      \end{enumerate}
		      \end{multicols}

		\item

		      Dos kilos de plátanos y tres de peras cuestan 8,80 soles.
		      Cinco kilos de plátanos y cuatro de peras cuestan 16,40 soles.
		      Determine el costo de kilo del plátano y de la pera según el siguiente requerimiento

		      \begin{multicols}{2}
			      \begin{enumerate}[a)]
				      \item

				            Modele el problema.

				      \item

				            Determine la norma matricial de $A$ y $A^{-1}$.

				      \item

				            Determine la solución usando los métodos de Gauss y
				            Gauss-Jordan.
			      \end{enumerate}
		      \end{multicols}

		\item

		      Se juntan 30 personas entre hombres, mujeres y niños.
		      Se sabe que entre los hombres y las mujeres duplican al
		      número de niños.
		      También se sabe que entre los hombres y el triple de las
		      mujeres exceden en 20 al doble de niños.

		      \begin{multicols}{2}
			      \begin{enumerate}[a)]
				      \item

				            Modele el problema.

				      \item

				            Determine la norma matricial de $A$ y $A^{-1}$.

				      \item


				            Determine la solución usando los métodos de Gauss y Gauss-Jordan.
			      \end{enumerate}
		      \end{multicols}
		      \saveenum
	\end{enumerate}
\end{frame}

\begin{frame}
	\begin{enumerate}
		\resume

		\item

		      Una editorial dispone de tres textos diferentes para
		      matemáticas de $2^{\circ}$ de secundaria.
		      El texto A se vende a 90 soles el ejemplar; el texto B a
		      110 soles y el $\mathrm{C}$ a 130 soles.
		      En la campaña correspondiente a un curso académico la
		      editorial ingresó, en concepto de ventas de estos libros de
		      matemáticas 84000 soles.

		      \begin{multicols}{2}
			      \begin{enumerate}[a)]
				      \item
				            Modele el problema.

				      \item

				            Determine la norma matricial de $A$ y $A^{-1}$.
				      \item

				            Determine la solución usando los métodos de Gauss y Gauss-Jordan.
			      \end{enumerate}
		      \end{multicols}
		      \saveenum
	\end{enumerate}
\end{frame}

\end{document}