%! Aldo Luna, Alexandra Mazzetti, Carlos Aznarán, Edward Canales
%! Universidad Nacional de Ingeniería
%! Facultad de Ciencias
%! Lima, Perú
%! Uso:
%! $ sudo pacman -Syu texlive-most zathura # dependencias, visor
%! $ arara primera
%! $ zathura primera.pdf
%! Ver https://wiki.archlinux.org/title/TeX_Live
% arara: clean: {
% arara: --> extensions:
% arara: --> ['aux','log','nav',
% arara: --> 'out','snm','toc','pytxcode','pdf']
% arara: --> }
% arara: lualatex: {
% arara: --> shell: yes,
% arara: --> draft: yes,
% arara: --> interaction: batchmode
% arara: --> }
% arara: biber
%! arara: pythontex
% arara: lualatex: {
% arara: --> shell: yes,
% arara: --> draft: yes,
% arara: --> interaction: batchmode
% arara: --> }
% arara: lualatex: {
% arara: --> shell: yes,
% arara: --> synctex: yes,
% arara: --> interaction: batchmode
% arara: --> }
% arara: clean: {
% arara: --> extensions:
% arara: --> ['aux','log','nav',
% arara: --> 'out','snm','toc','pytxcode']
% arara: --> }
\PassOptionsToPackage{svgnames}{xcolor}
\documentclass[
	spanish,
	8pt,
	utf8,
	xcolor=table,
	handout,
	aspectratio=169,
	professionalfonts,
	% notheorems,
	mathserif,
	leqno,
	% t
]{beamer}
\setbeamersize{text margin left=5pt,text margin right=5pt}
\usepackage[spanish,es-sloppy]{babel}
\spanishdatedel\decimalpoint
\usepackage{mathtools}
\usepackage{minted}
\usepackage{enumerate}
\usepackage{multicol}
% \usepackage{pythontex}
\usepackage[
	citestyle=numeric,
	style=apa,
	backend=biber,
	defernumbers=true,
	sorting=ynt,
	maxcitenames=4
]{biblatex}
\addbibresource{references.bib}

\newcolumntype{x}[1]{>{\centering\arraybackslash\hspace{0pt}}p{#1}}

\newcounter{savedenum}
\newcommand*{\saveenum}{\setcounter{savedenum}{\theenumi}}
\newcommand*{\resume}{\setcounter{enumi}{\thesavedenum}}

\setbeamertemplate{navigation symbols}{}
\setbeamertemplate{footline}{}
\setbeamertemplate{headline}{}

% https://tex.stackexchange.com/questions/68080/beamer-bibliography-icon
\setbeamertemplate{bibliography item}{%
	\ifboolexpr{ test {\ifentrytype{book}} or test {\ifentrytype{mvbook}}
		or test {\ifentrytype{collection}} or test {\ifentrytype{mvcollection}}
		or test {\ifentrytype{reference}} or test {\ifentrytype{mvreference}} }
	{\setbeamertemplate{bibliography item}[book]}
	{\ifentrytype{online}
		{\setbeamertemplate{bibliography item}[online]}
		{\setbeamertemplate{bibliography item}[article]}}%
	\usebeamertemplate{bibliography item}}

\defbibenvironment{bibliography}
{\list{}
	{\settowidth{\labelwidth}{\usebeamertemplate{bibliography item}}%
		\setlength{\leftmargin}{\labelwidth}%
		\setlength{\labelsep}{\biblabelsep}%
		\addtolength{\leftmargin}{\labelsep}%
		\setlength{\itemsep}{\bibitemsep}%
		\setlength{\parsep}{\bibparsep}}}
{\endlist}
{\item}

\title{
	\huge\sffamily
	Primera Práctica Dirigida\quad Grupo N$^{\circ}$~1
}

\subtitle{
	\large\scshape
	Análisis y Modelamiento Numérico I\quad CM4F1 B\\[.5\baselineskip]
		\normalsize\normalfont
		Profesor Jonathan Munguia La Cotera
}

\author{
	Aldo Luna Bueno\quad\and\quad
	Alexandra Gutierrez Mazzetti\quad\and\quad
	Carlos Aznarán Laos\quad\and\quad
	Edward Canales Yarin
}

\institute{\large
	Facultad de Ciencias \and
	Universidad Nacional de Ingeniería
}
\date{28 de septiembre del 2022}

\begin{document}

\frame{\titlepage}

% \begin{frame}
% 	\begin{theorem}[El Principio de Inducción Matemática]
% 		Sea $F$ un \alert{cuerpo ordenado}.
% 		Suponga que $\forall n\in\mathbb{N}_{F}$, $p\left(n\right)$ es
% 		una proposición acerca de $n$.
% 		Si

% 		\begin{multicols}{2}
% 			\begin{enumerate}[(1)]
% 				\item\label{hyp:1}

% 				$p\left(1\right)$ es verdadero, y

% 				\item\label{hyp:2}

% 				$\forall k\in\mathbb{N}_{F}$,
% 				$p\left(k\right)\implies p\left(k+1\right)$,
% 			\end{enumerate}
% 		\end{multicols}

% 		entonces $\forall n\in\mathbb{N}_{F}$, $p\left(n\right)$ es
% 		verdadero.
% 	\end{theorem}

% 	\begin{proof}
% 		Suponga que $p\left(n\right)$ es como se describe en la
% 		hipótesis.
% 		Sea
% 		\begin{math}
% 			A=
% 			\left\{
% 			x\in\mathbb{N}_{F}:
% 			p\left(x\right)\text{ es verdadero}
% 			\right\}
% 		\end{math}.
% 		Entonces,
% 		\begin{enumerate}[(i)]
% 			\item

% 			      $1\in A$, por~\eqref{hyp:1}.

% 			\item

% 			      Suponga que $x\in A$. Entonces, $x\in\mathbb{N}_{F}$ y
% 			      $p\left(x\right)$ es verdadero.
% 			      Así, por~\eqref{hyp:2}, $p\left(x+1\right)$ es verdadero.
% 			      Esto es, $x+1\in A$.
% 			      Por lo tanto, $x\in A\implies x+1\in A$.
% 			      Finalmente, $A$ es conjunto inductivo y
% 			      $\mathbb{N}_{F}\subset A$.
% 			      Esto es, $\forall n\in\mathbb{N}_{F}$, $p\left(n\right)$
% 			      es verdadero.
% 		\end{enumerate}
% 	\end{proof}

% 	\begin{definition}[Sucesión acotada]
% 		Una sucesión $\left\{x_{n}\right\}$ es \alert{acotada} si el
% 		conjunto $\left\{x_{n}:n\in\mathbb{N}\right\}$ es un conjunto
% 		acotado.
% 		Hay dos maneras equivalentes de decir que $\left\{x_{n}\right\}$
% 		es \alert{acotada}:
% 		\begin{multicols}{2}
% 			\begin{enumerate}[(1)]
% 				\item

% 				      $\exists a,b\in\mathbb{R}$ tal que
% 				      $\forall n\in\mathbb{N}$, $a\leq x_{n}\leq b$.

% 				\item

% 				      $\exists M>0$ tal que $\forall n\in\mathbb{N}$,
% 				      $\left|x_{n}\right|\leq M$.
% 			\end{enumerate}
% 		\end{multicols}
% 	\end{definition}
% \end{frame}

% \begin{frame}
% 	\begin{definition}[Sucesiones monótonas]
% 		Una sucesión $\left\{a_{n}\right\}$ es
% 		\begin{enumerate}[a)]
% 			\item

% 			      \alert{monótona creciente} sii $\forall n\in\mathbb{N}$,
% 			      $a_{n}\leq a_{n+1}$; esto es

% 			      \begin{equation*}
% 				      a_{1}\leq
% 				      a_{2}\leq
% 				      \cdots\leq
% 				      a_{n}\leq
% 				      a_{n+1}\leq
% 				      \cdots.
% 			      \end{equation*}

% 			\item

% 			      \alert{monótona decreciente} sii
% 			      $\forall n\in\mathbb{N}$, $a_{n}\geq a_{n+1}$; esto es

% 			      \begin{equation*}
% 				      a_{1}\geq
% 				      a_{2}\geq
% 				      \cdots\geq
% 				      a_{n}\geq
% 				      a_{n+1}\geq
% 				      \cdots.
% 			      \end{equation*}

% 			\item

% 			      \alert{estrictamente creciente} sii
% 			      $\forall n\in\mathbb{N}$, $a_{n}<a_{n+1}$; esto es

% 			      \begin{equation*}
% 				      a_{1}<
% 				      a_{2}<
% 				      \cdots<
% 				      a_{n}<
% 				      a_{n+1}<
% 				      \cdots.
% 			      \end{equation*}

% 			\item

% 			      \alert{estrictamente decreciente} sii $\forall n\in\mathbb{N}$,
% 			      $a_{n}>a_{n+1}$; esto es

% 			      \begin{equation*}
% 				      a_{1}>
% 				      a_{2}>
% 				      \cdots>
% 				      a_{n}>
% 				      a_{n+1}>
% 				      \cdots.
% 			      \end{equation*}
% 		\end{enumerate}
% 	\end{definition}

% \end{frame}

% \begin{frame}
% 	\begin{theorem}[Convergencia Monótona para sucesiones]
% 		Cualquier sucesión monótona y acotada \alert{converge}.
% 		Más precisamente, si $\left\{a_{n}\right\}$ es una sucesión
% 		monótona creciente que es acotada superiormente, entonces el
% 		\begin{math}
% 			\lim\limits_{n\to\infty}a_{n}=
% 			\sup\left\{a_{n}:n\in\mathbb{N}\right\}
% 		\end{math}.

% 		% \begin{enumerate}[a)]
% 		% 	\item
% 		% \item
% 		% Si $\left\{a_{n}\right\}_{n\in\mathbb{N}}$ es una sucesión
% 		% monótona decreciente que es acotada inferiormente, entonces el
% 		% $\lim\limits_{n\to\infty}a_{n}=\inf\left\{a_{n}:n\in\mathbb{N}\right\}$;
% 		% \end{enumerate}
% 	\end{theorem}

% 	\begin{proof}
% 		% \begin{enumerate}[a)]
% 		% 	\item

% 		Suponga que $\left\{a_{n}\right\}_{n\in\mathbb{N}}$ es acotada y
% 		monótona creciente.
% 		Dado que $\left\{a_{n}\right\}_{n\in\mathbb{N}}$ es acotada, el
% 		conjunto $\left\{a_{n}:n\in\mathbb{N}\right\}$ tiene una cota
% 		superior.
% 		Por la completitud de $\mathbb{R}$,
% 		$\exists u=\sup\left\{a_{n}:n\in\mathbb{N}\right\}$.

% 		Sea $\varepsilon>0$.
% 		Por el criterio $\varepsilon$ para el supremo,
% 		$\exists n_{0}\in\mathbb{N}$ tal que $a_{n_{0}}>u-\varepsilon$.
% 		Pero $\left\{a_{n}\right\}$ es monótona creciente, por lo tanto,
% 		$n\geq n_{0}\implies a_{n}\geq a_{n_{0}}$.

% 		Así,
% 		\begin{equation}\label{eq:increasing}
% 			n\geq n_{0}\implies a_{n}\geq a_{n_{0}}>u-\varepsilon.
% 		\end{equation}

% 		Pero, dado que $u=\sup\left\{a_{n}:n\in\mathbb{N}\right\}$,
% 		\begin{equation}\label{eq:upperbound}
% 			\forall n\in\mathbb{N}, a_{n}\leq u.
% 		\end{equation}
% 		Juntando~\eqref{eq:increasing} y~\eqref{eq:upperbound}, tenemos
% 		\begin{align*}
% 			n\geq n_{0}
% 			 & \implies u-\varepsilon<a_{n}\leq u< u+\varepsilon \\
% 			 & \implies u-\varepsilon<a_{n}<u+\varepsilon        \\
% 			 & \implies \left|a_{n}-u\right|<\varepsilon.
% 		\end{align*}
% 		Por lo tanto,
% 		\begin{math}
% 			\lim\limits_{n\to\infty}a_{n}=
% 			u=
% 			\sup\left\{a_{n}:n\in\mathbb{N}\right\}
% 		\end{math}.
% 		% \item De manera análoga.
% 		% \end{enumerate}
% 	\end{proof}

% 	% \begin{corollary}[Teorema Fundamental de las Sucesiones Monótonas]
% 	% 	Una sucesión monótona converge sii esta es es acotada.
% 	% \end{corollary}

% 	% \begin{theorem}[Teorema de Bolzano-Weierstraß para sucesiones]
% 	% 	\label{thm:bolzano-weierstraß}
% 	% 	Cualquier sucesión acotada tiene una subsucesión convergente.
% 	% \end{theorem}

% 	% \begin{proof}
% 	% 	\begin{enumerate}
% 	% 		\item

% 	% 		      Suponga cualquier sucesión acotada.

% 	% 		\item

% 	% 		      Cree el intervalo cerrado inicial cuyos extremos son
% 	% 					dichas cotas y construya una sucesión de intervalos
% 	% 					encajados.

% 	% 		\item

% 	% 		      Use el teorema de intervalos encajados de Cantor y
% 	% 					concluya que existe un único número real en la
% 	% 					intersección.

% 	% 		\item

% 	% 		      Pruebe que la subsucesión converge a ese número real.
% 	% 	\end{enumerate}
% 	% \end{proof}

% 	% \begin{theorem}
% 	% 	Una sucesión acotada es convergente sii tiene uno y solo un punto
% 	% 	de acumulación.
% 	% \end{theorem}

% 	% \begin{proof}
% 	% 	Para el recíproco use el teorema~\ref{thm:bolzano-weierstraß} y
% 	% 	por contradicción.
% 	% \end{proof}
% \end{frame}

% \begin{frame}
% 	\begin{definition}[Funciones monótonas]
% 		Una función $f$ es
% 		\begin{enumerate}[(a)]
% 			\item\label{mon:increasing}

% 			\alert{monótona creciente} en un conjunto
% 			$A\subset\mathcal{D}\left(f\right)$ sii
% 			$\forall x_{1},x_{2}$ en $A$,

% 			\begin{equation*}
% 				x_{1}<
% 				x_{2}\implies
% 				f\left(x_{1}\right)\leq
% 				f\left(x_{2}\right);
% 			\end{equation*}

% 			\item\label{mon:decreasing}

% 			\alert{monótona decreciente} en un conjunto
% 			$A\subset\mathcal{D}\left(f\right)$ sii
% 			$\forall x_{1},x_{2}$ en $A$,

% 			\begin{equation*}
% 				x_{1}<
% 				x_{2}\implies
% 				f\left(x_{1}\right)\geq
% 				f\left(x_{2}\right);
% 			\end{equation*}

% 			\item\label{mon:strict-increasing}

% 			\alert{estrictamente creciente} en un conjunto
% 			$A\subset\mathcal{D}\left(f\right)$ sii
% 			$\forall x_{1},x_{2}$ en $A$,

% 			\begin{equation*}
% 				x_{1}<
% 				x_{2}\implies
% 				f\left(x_{1}\right)<
% 				f\left(x_{2}\right);
% 			\end{equation*}

% 			\item\label{mon:strict-decreasing}

% 			\alert{estrictamente decreciente} en un conjunto
% 			$A\subset\mathcal{D}\left(f\right)$ sii
% 			$\forall x_{1},x_{2}$ en $A$,

% 			\begin{equation*}
% 				x_{1}<
% 				x_{2}\implies
% 				f\left(x_{1}\right)>
% 				f\left(x_{2}\right);
% 			\end{equation*}

% 			\item

% 			      \alert{monótona} en $A\subset\mathcal{D}\left(f\right)$
% 			      si este satisface~\eqref{mon:increasing}
% 			      o~\eqref{mon:decreasing}
% 			      y \alert{estrictamente monótona} en
% 			      $A\subset\mathcal{D}\left(f\right)$ si este satisface
% 			      satisface~\eqref{mon:strict-increasing}
% 			      o~\eqref{mon:strict-decreasing}.
% 		\end{enumerate}
% 	\end{definition}

% \end{frame}

% \begin{frame}
% 	\begin{theorem}
% 		Suponga que $f$ es diferenciable sobre un intervalo $I$.
% 		\begin{enumerate}[(a)]
% 			\item\label{crit:increasing}

% 			Si $f^{\prime}\left(x\right)\geq 0$, $\forall x\in I$, entonces
% 			$f$ es monótona creciente sobre $I$.

% 			\item\label{crit:decreasing}

% 			Si $f^{\prime}\left(x\right)\leq 0$, $\forall x\in I$, entonces
% 			$f$ es monótona decreciente sobre $I$.

% 			\item\label{crit:strict-increasing}

% 			Si $f^{\prime}\left(x\right)>0$, $\forall x\in I$, entonces
% 			$f$ es estrictamente creciente sobre $I$.

% 			\item\label{crit:strict-decreasing}

% 			Si $f^{\prime}\left(x\right)<0$, $\forall x\in I$, entonces
% 			$f$ es estrictamente decreciente sobre $I$.
% 		\end{enumerate}
% 	\end{theorem}

% 	\begin{alertblock}{Tener cuidado}
% 		Las reciprocas de~\eqref{crit:strict-increasing}
% 		y~\eqref{crit:strict-decreasing} son falsas.
% 	\end{alertblock}

% 	\begin{proof}
% 		Suponga que $f$ es diferenciable sobre un intervalo $I$.
% 		\begin{enumerate}[(a)]
% 			\item

% 			      Suponga que $f^{\prime}\left(x\right)\geq0$,
% 			      $\forall x\in I$.
% 			      Sea $x_{1}<x_{2}$ en $I$.
% 			      Aplicando el \alert{teorema del valor medio} a $f$ en el
% 			      intervalo $\left[x_{1},x_{2}\right]$,
% 			      $\exists c\in\left(x_{1},x_{2}\right)$ tal que
% 			      \begin{math}
% 				      f^{\prime}\left(c\right)=
% 				      \frac{
% 				      f\left(x_{2}\right)-f\left(x_{1}\right)
% 				      }{
% 				      x_{2}-x_{1}
% 				      }
% 			      \end{math}.
% 			      Dado que $f^{\prime}\geq0$,
% 			      \begin{equation*}
% 				      \frac{
% 				      f\left(x_{2}\right)-
% 				      f\left(x_{1}\right)
% 				      }{x_{2}-x_{1}}\geq
% 				      0.
% 			      \end{equation*}
% 			      Pero $x_{2}-x_{1}>0$ dado que $x_{1}<x_{2}$.
% 			      Así, $f\left(x_{2}\right)-f\left(x_{1}\right)\geq0$;
% 			      esto es,
% 			      \begin{math}
% 				      f\left(x_{2}\right)\geq
% 				      f\left(x_{1}\right).
% 			      \end{math}
% 			      Hemos provado que $\forall x_{1}<x_{2}$ en $I$,
% 			      $f\left(x_{2}\right)\geq f\left(x_{1}\right)$.
% 			      Esto es, $f$ es monótona creciente sobre $I$.
% 		\end{enumerate}
% 	\end{proof}
% \end{frame}

% \begin{frame}[fragile]
% 	\printpythontex
% \end{frame}

\begin{frame}

	\begin{enumerate}\setcounter{enumi}{4}
		\item

		      Sea $\left\{a_{n}\right\}_{n\in\mathbb{N}}$ la sucesión
		      definida por

		      \begin{equation*}
			      a_{1}=\sqrt{3},\quad
			      a_{n}=\sqrt{3a_{n-1}}.
		      \end{equation*}

		      Determine $\lim\limits_{n\to\infty}a_{n}$.
	\end{enumerate}

	\begin{solution}
		Primero probaremos que la \alert{existencia del límite} y luego
		lo calcularemos.
		Según el teorema de convergencia monótona, si la sucesión
		$a_{n}$ es monótona y acotada, entonces converge a un valor
		(el límite que se pide determinar).

		\begin{description}
			\item[$a_{n}$ es monótona creciente.]

				Definimos la proposición $p\left(n\right)\coloneqq a_{n+1}\geq a_{n}$.

				Por el \alert{Principio de Inducción Matemática}, vemos que
				el caso base $p\left(1\right)$ se cumple
				\begin{equation*}
					a_{2}=\sqrt{3\sqrt3}>
					\sqrt{3}=a_{1}.
				\end{equation*}

				Para la hipótesis inductiva, asumamos que $p\left(k\right)$ se
				cumple hasta un $k\in\mathbb{N}$ fijo.
				\begin{align*}
					a_{k+1}         & \geq a_{k}        \\
					3a_{k+1}        & \geq 3a_{k}       \\
					\sqrt{3a_{k+1}} & \geq\sqrt{3a_{k}} \\
					a_{k+2}         & \geq a_{k+1}.
				\end{align*}

				Luego, $p\left(k+1\right)$ se cumple.
				Por lo tanto, $p\left(n\right)$ se cumple para todo $n$ natural
				y la sucesión $a_{n}$ es monótona creciente.

		\end{description}
	\end{solution}
\end{frame}

\begin{frame}
	\begin{solution}
		\begin{description}
			\item[$a_{n}$ es acotada superiormente.]

				Sea la proposición $q\left(n\right)\coloneqq\text{existe un natural $M$ tal que $a_n\leq M$}$.

				Por el \alert{Principio de Inducción Matemática}, si $M=3$ vemos que el caso base
				$q\left(1\right)$ se cumple
				\begin{equation*}
					a_{1}=\sqrt{3}\leq 3.
				\end{equation*}

				Para la hipótesis inductivo, asumamos que $q\left(k\right)$ se
				cumple hasta un $k\in\mathbb{N}$ fijo.

				\begin{align*}
					a_{k+1}       & \leq a_{k}   \\
					a_{k}         & \leq 3       \\
					3a_{k}        & \leq 9       \\
					\sqrt{3a_{k}} & \leq\sqrt{9} \\
					a_{k+1}       & \leq 3.
				\end{align*}

				Luego, $q\left(k+1\right)$ se cumple.
				Por lo tanto, $q\left(n\right)$ se cumple para todo $n$ natural
				y la sucesión $a_{n}$ es acotada superiormente.

				\alert{Cálculo del límite.}
				\begin{equation*}
					L=\lim\limits_{n\to\infty}a_{n}=
					\lim\limits_{n\to\infty}\sqrt{3a_{n-1}}=
					\sqrt{3\lim\limits_{n\to\infty}a_{n-1}}=
					\sqrt{3L}
					\implies
					L\left(L-3\right)=0
					\implies
					L=0 \vee L=3.
				\end{equation*}
				Sabemos que $a_{n}$ es una sucesión monótona creciente, así
				que $a_{n}\leq a_{1}=\sqrt{3}$ para todo $n\geq1$
				Luego, $L\geq\sqrt{3}>0$, por lo que $L=3$.
		\end{description}
	\end{solution}
\end{frame}

% \begin{frame}
% 	\begin{solution}
% 		Otra forma.

% 		\begin{description}[leftmargin=*] %\leavevmode
% 			\item[$a_{n}$ es estrictamente creciente.]

% 				Definimos la proposición $p\left(n\right)$:

% 				Definimos la función
% 				\begin{align*}
% 					f\colon\mathbb{R}^{+} & \longrightarrow\mathbb{R} \\
% 					t                     & \longmapsto\sqrt{3t}
% 				\end{align*}
% 				Además,
% 				\begin{math}
% 					t>0\implies
% 					3t>0\implies
% 					f\left(t\right)=\sqrt{3t}>0\implies
% 					\frac{1}{f\left(t\right)}>0\implies
% 					f^{\prime}\left(t\right)=\frac{3}{2f\left(t\right)}>0.
% 				\end{math}
% 				Por lo tanto, $f$ es estrictamente creciente en
% 				$\mathbb{R}^{+}$.
% 				Y como
% 				\begin{math}
% 					\left\{a_{n}\right\}_{n\in\mathbb{N}}\subset
% 					f\left(\mathbb{R}^{+}\right)
% 				\end{math}.
% 				Concluimos que $\left\{a_{n}\right\}_{n\in\mathbb{N}}$ es
% 				estrictamente creciente.

% 			\item[$a_{n}$ es acotada superiormente.]

% 				% $\exists 3>0$ tal que $\forall n\in\mathbb{N}$,
% 				% $a_{n}=\sqrt{3a_{n-1}}\leq 3$.
% 				% acotada superiormente
% 				% $\lim\limits_{n\to\infty}\sqrt[n]{x}=1$, $\forall x>0$.
% 				% teorema del punto fijo de Banach.

% 				Supongamos que $p\left(n\right)$ es la proposición
% 				$\exists 3>0$ tal que $a_{n}\leq 3$.
% 				Por el \alert{Principio de Inducción Matemática},

% 				\begin{enumerate}[(1)]
% 					\item\label{hyp:5.1}

% 					$p\left(1\right)$ es verdadero, es decir, $\exists 3>0$
% 					tal que $a_{1}=\sqrt{3}\leq 3$.

% 					\item\label{hyp:5.2}

% 					$\forall k\in\mathbb{N}$, $p\left(k\right)$ es verdadero,
% 					es decir, $\exists 3>0$ tal que
% 					$a_{k}=\sqrt{3a_{k-1}}\leq 3$.
% 				\end{enumerate}

% 				Veamos que $p\left(k+1\right)$ es verdadero.
% 				En efecto, por la hipótesis de inducción~\eqref{hyp:5.2},
% 				$p\left(k\right)$ es verdadero, es decir,
% 				$\exists 3>0$ tal que
% 				\begin{math}
% 					a_{k}\leq 3\implies
% 					3a_{k}\leq 3\times 3\implies
% 					\sqrt{3a_{k}}\leq\sqrt{9}\implies
% 					a_{k+1}\leq3
% 				\end{math}.
% 				En otras palabras, $p\left(k+1\right)$ es verdadero.
% 				Por lo tanto, $\forall n\in\mathbb{N}$, $\exists 3>0$
% 				tal que $a_{n}\leq 3$.
% 		\end{description}
% 	\end{solution}
% \end{frame}

% \begin{frame}
% 	Como $\left\{a_{n}\right\}$ es monótona creciente y acotada
% 	superiormente, entonces el
% 	\begin{math}
% 		\lim\limits_{n\to\infty}
% 		\left\{a_{n}\right\}_{n\in\mathbb{N}}=
% 		L=
% 		\sup\left\{a_{n}:n\in\mathbb{N}\right\}
% 	\end{math}.

% 	O sea, $\exists L\in\mathbb{R}$ tal que
% 	\begin{align*}
% 		\lim\limits_{n\to\infty}a_{n}           & =L.                 \qquad\text{Ya que la sucesión $a_{n}$ es convergente.}          & \\
% 		\lim\limits_{n\to\infty}\sqrt{3a_{n-1}} & =L.                 \qquad\text{Definición de $a_{n}$.}                              & \\
% 		\sqrt{\lim\limits_{n\to\infty}3a_{n-1}} & =L.                 \qquad\text{La raíz cuadrada es una función continua.}           & \\
% 		\sqrt{3\lim\limits_{n\to\infty}a_{n-1}} & =L.                                                                                  & \\
% 		\sqrt{3L}                               & =L.              \qquad\text{Ya que la sucesión $a_{n}$ es convergente.}             & \\
% 		3L                                      & =L^{2}.                                                                              & \\
% 		0                                       & =L\left(L-3\right). \qquad\text{$L\neq0$ porque $a_{n}$ es estrictamente creciente.} & \\
% 		3                                       & =L.                                                                                  & \\
% 	\end{align*}
% \end{frame}

% \begin{frame}
% 	\begin{theorem}[Aceleración de Aitken]
% 		Sea $r_{n}$ una sucesión de números reales que converge al límite
% 		$r$.
% 		Entonces, la nueva sucesión
% 		\begin{equation*}
% 			s_{n}=
% 			\frac{r_{n}r_{n+2}-r^{2}_{n+1}}{r_{n+2}-2r_{n+1}+r_{n}}\quad
% 			\left(n\geq0\right)
% 		\end{equation*}
% 		converge a $r$ más rápido, si
% 		$r_{n+1}-r=\left(c+\delta_{n}\right)\left(r_{n}-r\right)$ con
% 		$\left|c\right|<1$
% 		y $\lim\limits_{n\to\infty}\delta_{n}=0$.
% 		De hecho, $\lim\limits_{n\to\infty}\frac{s_{n}-r}{r_{n}-r}=0$.
% 	\end{theorem}

% 	\begin{proof}
% 		Defina el error de la sucesión como $h_{n}\coloneqq r_{n}-r$. Entonces,
% 		\begin{align*}
% 			s_{n}   & =
% 			\frac{
% 			\left(r+h_{n}\right)
% 			\left(r+h_{n+2}\right)-
% 			{\left(r+h_{n+1}\right)}^2
% 			}{
% 			\left(r+h_{n+2}\right)-
% 			2\left(r+h_{n+1}\right)+
% 			\left(r+h_{n}\right)
% 			}             \\
% 			        & =r+
% 			\frac{h_{n}h_{n+2}-h^{2}_{n+1}}{h_{n+2}-2h_{n+1}+h_{n}}
% 			\shortintertext{
% 			Use la hipótesis $h_{n+1}=\left(c+\delta_{n}\right)h_{n}$
% 			para obtener
% 			\begin{math}
% 				h_{n+2}=
% 				\left(c+\delta_{n+1}\right)
% 				\left(c+\delta_{n}\right)h_{n}
% 			\end{math}
% 			y
% 			}
% 			s_{n}-r & =
% 			\frac{
% 			h_{n}
% 			\left(c+\delta_{n+1}\right)
% 			\left(c+\delta_{n}\right)
% 			h_{n}-
% 			{\left(c+\delta_{n}\right)}^{2}h^{2}_{n}
% 			}{
% 			\left(c+\delta_{n+1}\right)
% 			\left(c+\delta_{n}\right)
% 			h_{n}-
% 			2\left(c+\delta_{n}\right)
% 			h_{n}+
% 			h_{n}
% 			}             \\
% 			        & =
% 			h_{n}
% 			\frac{
% 			\left(c+\delta_{n+1}\right)
% 			\left(c+\delta_{n}\right)-
% 			{\left(c+\delta_{n}\right)}^{2}
% 			}{
% 			\left(c+\delta_{n+1}\right)
% 			\left(c+\delta_{n}\right)-
% 			2\left(c+\delta_{n}\right)+
% 			1
% 			}
% 		\end{align*}
% 		Y el $\lim\limits_{n\to\infty}\frac{s_{n}-r}{h_{n}}=0$.
% 	\end{proof}
% \end{frame}

% \begin{frame}
% 	\begin{definition}[Orden de convergencia]
% 		Sea $x_{n}$ una sucesión de números reales que tiene al límite $x^{\ast}$.
% 		Decimos que la tasa de convergencia es de por lo menos
% 		\begin{itemize}
% 			\item

% 			      \alert{lineal} sii

% 			      \begin{equation*}
% 				      \exists c<1, N\in\mathbb{N}\text{ tal que }
% 				      \forall n\geq N:
% 				      \left|x_{n+1}-x^{\ast}\right|\leq
% 				      c\left|x_{n}-x^{\ast}\right|.
% 			      \end{equation*}

% 			\item

% 			      \alert{superlineal} sii

% 			      \begin{equation*}
% 				      \exists \left\{\varepsilon_{n}\right\}\searrow0, N\in\mathbb{N}\text{ tal que }
% 				      \forall n\geq N:
% 				      \left|x_{n+1}-x^{\ast}\right|\leq
% 				      \varepsilon_{n}\left|x_{n}-x^{\ast}\right|.
% 			      \end{equation*}

% 			\item

% 			      \alert{cuadrática} sii

% 			      \begin{equation*}
% 				      \exists c\in\mathbb{R}, N\in\mathbb{N}\text{ tal que }
% 				      \forall n\geq N:
% 				      \left|x_{n+1}-x^{\ast}\right|\leq
% 				      c{\left|x_{n}-x^{\ast}\right|}^2.
% 			      \end{equation*}

% 			\item

% 			      \alert{orden $\alpha$} sii

% 			      \begin{equation*}
% 				      \exists c\in\mathbb{R}, N\in\mathbb{N}\text{ tal que }
% 				      \forall n\geq N:
% 				      \left|x_{n+1}-x^{\ast}\right|\leq
% 				      c{\left|x_{n}-x^{\ast}\right|}^{\alpha}.
% 			      \end{equation*}
% 		\end{itemize}

% 	\end{definition}
% \end{frame}

\begin{frame}
	\begin{enumerate}\setcounter{enumi}{6}
		\item

		      Sea la sucesión $\left\{x_{n}\right\}_{n\in\mathbb{N}}$
		      definida por

		      \begin{equation*}
			      x_{n}=\frac{1}{2^{n}}.
		      \end{equation*}

		      Determine los valores de la sucesión usando el método
		      $\Delta^{2}$ de Aitken.

	\end{enumerate}

	\begin{solution}
		Debemos asegurar que $x_{n}$ es convergente y al menos del tipo lineal.
		\begin{description}
			\item[$x_{n}$ converge a $0$.]

				\begin{equation*}
					x_{n}=\frac{1}{2^{n}}.
				\end{equation*}
				Vemos que:
				\begin{math}
					\lim\limits_{n\to\infty}
					{
						x_{n}
					}=\lim\limits_{n\to\infty}
					\dfrac{
						{1}
					}{
						{{2^{n}}}
					}=0
				\end{math}.
				% \begin{math}
				% 	\forall n\in\mathbb{N}:
				% 	n<2^{n}\iff
				% 	0<
				% 	\frac{1}{2^{n}}<
				% 	\frac{1}{n}
				% \end{math}.
				% Por el teorema del encaje, tenemos que el
				% $\lim\limits_{n\to\infty}x_{n}=0$.

				% \begin{math}
				% 	\forall\varepsilon>0:
				% 	\exists n_{0}=
				% 	\left\lceil
				% 	-\log_{2}\varepsilon
				% 	\right\rceil
				% 	\in\mathbb{N}
				% \end{math}
				% tal que
				% \begin{math}
				% 	\forall n\geq n_{0}:
				% 	\left|\frac{1}{2^{n}}-0\right|=
				% 	\left|\frac{1}{2^{n}}\right|=
				% 	\frac{1}{2^{n}}<
				% 	\varepsilon
				% \end{math}.

			\item[$x_{n}$ converge linealmente.]

				\begin{align*}
					\left|x_{n+1}-L\right| =
					\left|\frac{1}{2^{n+1}}\right|
				\end{align*}

				\begin{math}
					\lim\limits_{n\to\infty}
					\dfrac{
					\left|x_{n+1}-L\right|
					}{
					{\left|x_{n}-L\right|}^{1}
					}=
					\lim\limits_{n\to\infty}
					\dfrac{
					\left|{2^{-n-1}}-0\right|
					}{
					{\left|{2^{-n}}-0\right|}^{1}
					}
					=\lim\limits_{n\to\infty}
					{
						{2^{-1}}
					}
					=
					\dfrac{1}{2}.
				\end{math}

				% \begin{align*}
				% 	\exists c<1,N\in\mathbb{N}\text{ tal que }\forall n\geq N:
				% 	\left|x_{n+1}-x^{\ast}\right|    & \leq
				% 	c\left|x_{n}-x^{\ast}\right|.           \\
				% 	\exists \frac{1}{2}<1,1\in\mathbb{N}\text{ tal que }\forall n\geq 1:
				% 	\left|\frac{1}{2^{n+1}}-0\right| & \leq
				% 	\frac{1}{2}\left|\frac{1}{2^{n}}-0\right|.
				% \end{align*}

				% \begin{math}
				% 	\lim\limits_{n\to\infty}
				% 	\frac{
				% 	\left|x_{n+1}-x^{\ast}\right|
				% 	}{
				% 	{\left|x_{k}-L\right|}^{1}
				% 	}=
				% 	\lim\limits_{n\to\infty}
				% 	\frac{
				% 		\left|\frac{1}{2^{n+1}}-0\right|
				% 	}{
				% 		{\left|\frac{1}{2^{n}}-0\right|}^{1}
				% 	}
				% 	=\lim\limits_{n\to\infty}=
				% 	\frac{
				% 		\left|\frac{1}{2^{n}}\cdot\frac{1}{2}\right|
				% 	}{
				% 		{\left|\frac{1}{2^{n}}\right|}^{1}
				% 	}
				% 	=
				% 	\frac{1}{2}\in\mathbb{R}.
				% \end{math}

				% \begin{align*}
				% 	r_{n+1}-r           & =\left(c+\delta_{n}\right)\left(r_{n}-r\right)                     \\
				% 	\frac{1}{2^{n+1}}-0 & =\left(\frac{1}{2}+\delta_{n}\right)\left(\frac{1}{2^{n}}-0\right) \\
				% 	0                   & = \delta_{n},\quad\lim\limits_{n\to\infty}\delta_{n}=0.
				% \end{align*}
		\end{description}
	\end{solution}
\end{frame}

\begin{frame}
	\begin{solution}
		Por el método $\Delta^{2}$ de Aitken, tenemos
		\begin{align*}
			\hat{x}_{n} & =
			x_{n}-
			\frac{{\left(x_{n+1}-x_{n}\right)}^{2}}{x_{n+2}-2x_{n+1}+x_{n}} \\
			            & ={2^{-n}}-
			\frac{{\left({2^{-n-1}}-{2^{-n}}\right)}^{2}}{{2^{-n-2}}-{2{\left({2^{-n-1}}\right)}}+{2^{-n}}}
			\\
			            & =0.
		\end{align*}
		% \begin{align*}
		% 	\hat{p}_{n}=
		% 	p_{n}-
		% 	\frac{
		% 	{\left(p_{n+1}-p_{n}\right)}^{2}
		% 	}{
		% 	p_{n+2}-2p_{n+1}+p_{n}
		% 	}
		% 	=\frac{1}{2^{n}}-
		% 	\frac{
		% 		{\left(
		% 				\frac{1}{2^{n+1}}-\frac{1}{2^{n}}
		% 				\right)}^{2}
		% 	}{
		% 		\frac{1}{2^{n+2}}-\frac{2^{1}}{2^{n+1}}+\frac{1}{2^{n}}
		% 	}
		% 	 & =\frac{1}{2^{n}}-
		% 	\frac{
		% 		\frac{1}{2^{2n+2}}-\frac{2^{1}}{2^{n+1}\cdot2^{n}}+\frac{1}{2^{2n}}
		% 	}{
		% 		\frac{1}{2^{n+2}}-\frac{1}{2^{n}}+\frac{1}{2^{n}}
		% 	}                    \\
		% 	 & =\frac{1}{2^{n}}-
		% 	\frac{
		% 		\frac{1}{2^{2n+2}}-\frac{1}{2^{2n}}+\frac{1}{2^{2n}}
		% 	}{
		% 		\frac{1}{2^{n+2}}
		% 	}                    \\
		% 	 & =\frac{1}{2^{n}}-
		% 	\frac{
		% 		\frac{1}{2^{2n+2}}
		% 	}{
		% 		\frac{1}{2^{n+2}}
		% 	}                    \\
		% 	 & =\frac{1}{2^{n}}-
		% 	\frac{
		% 		1
		% 	}{
		% 		2^{n}
		% 	}                    \\
		% 	 & =0.
		% \end{align*}

		\begin{table}[ht!]
			\centering
			\begin{tabular}{|x{0.5cm}x{1cm}x{0.5cm}|}
				\hline
				$n$      & $x_{n}$  & $\hat{x}_{n}$ \\
				\hline
				$1$      & $0.5000$ & $0$           \\
				% \hline
				$2$      & $0.2500$ & $0$           \\
				% \hline
				$3$      & $0.1250$ & $0$           \\
				% \hline
				$4$      & $0.0625$ & $0$           \\
				\hline
				$\infty$ & $0$      & $0$           \\
				\hline
			\end{tabular}
		\end{table}
	\end{solution}
\end{frame}

\begin{frame}
	\begin{enumerate}\setcounter{enumi}{10}
		\item

		      Implemente un programa en \texttt{python} para convertir
		      los siguientes números binarios con signo de $8$ bits a
		      base $10$.

		      \begin{multicols}{4}

			      \begin{enumerate}[a)]
				      \item

				            $\left(10111000\right)_{2}$.

				      \item

				            $\left(01010101\right)_{2}$.

				      \item

				            $\left(11111111\right)_{2}$.

				      \item

				            $\left(00011011\right)_{2}$.
			      \end{enumerate}
		      \end{multicols}

	\end{enumerate}

	\begin{solution}
		\begin{minipage}{0.55\textwidth}
			% \url{https://docs.python.org/3/library/functions.html\#int}.
			\begin{listing}[H]
				\inputminted[
					fontsize=\footnotesize,
					breaklines,
				]{python}{binary2decimal.py}
			\end{listing}
		\end{minipage}
		\begin{minipage}{0.35\textwidth}
			\begin{listing}[H]
				\inputminted[
					fontsize=\footnotesize,
				]{text}{output.txt}
			\end{listing}
		\end{minipage}
	\end{solution}
\end{frame}

\begin{frame}
	\begin{enumerate}\setcounter{enumi}{19}
		\item

		      Probar que el cardinal del sistema de números de punto
		      flotante normalizado
		      $\mathbb{F}\left(\beta, t, L, U\right)$ es

		      \begin{equation*}
			      \operatorname{card}\left(\mathbb{F}\right)=
			      2\left(\beta-1\right)\beta^{t-1}\left(U-L+1\right)+1.
		      \end{equation*}

		      En particular, para $\mathbb{F}\left(10,3,-2,3\right)$,
		      calcular $x_{\min}$, $x_{\max}$, $\epsilon_{M}$.

	\end{enumerate}
\end{frame}

\begin{frame}
	\begin{definition}[Representación de
			punto flotante]
		Un número real no nulo $x$, tiene como
		\alert{representación de punto flotante} a
		\begin{equation*}
			x=
			{\left(-1\right)}^{s}\cdot
			\left(0.a_{1}\ldots a_{t}\right)\cdot
			{\beta}^{e}=
			{\left(-1\right)}^{s}\cdot
			m\cdot{\beta}^{e-t},
		\end{equation*}
		donde $t\in\mathbb{N}$ es el número de dígitos significativos
		permitidos $a_{i}$ (con $0\leq a_{i}\leq\beta-1$),
		$m=a_{1}\ldots a_{t}$ es un número entero llamado mantisa tal
		que $0\leq m\leq{\beta}^{t}-1$ y $e$ es número entero
		llamado exponente, donde $0>L\leq e\leq U>0$.
	\end{definition}

	\begin{definition}[El conjunto de los números punto flotante]
		El conjunto de los números punto flotante con $t$ dígitos
		significativos, base $\beta\geq2$, $0\leq a_{i}\leq\beta-1$ y
		$L\leq e\leq U$.
		\begin{equation*}
			\mathbb{F}\left(\beta, t, L, U\right)=
			\left\{0\right\}\cup
			\left\{
			x\in\mathbb{R}:x=
			{\left(-1\right)}^{s}
			\beta^{e}
			\sum_{i=1}^{t}a_{i}\beta^{-i}
			\right\}.
		\end{equation*}
	\end{definition}
	Con el fin de tener una representación numérica única, asumimos
	que $a_{i}\neq0$ y $m\geq{\beta}^{t-1}$.
	$a_{1}$ es el dígito significativo principal, $a_{t}$ es el
	último dígito significativo y la
	representación de $x$ es llamado \alert{normalizado}.
\end{frame}

\begin{frame}
	\begin{solution}
		Sea el sistema de punto flotante normalizado, un elemento de
		$\mathbb{F}\left(\beta, t, L, U\right)$ es de la forma
		\begin{equation*}
			x=
			{\left(-1\right)}^{s}\cdot
			\left(0.a_{1}\ldots a_{t}\right)\cdot
			{\beta}^{e}.
		\end{equation*}

		Notamos que

		\begin{itemize}
			\item

			      El signo $s\in\left\{0,1\right\}$ toma dos valores.

			\item

			      Los dígitos significativos
			      $a_{i}\in\left\{0,\ldots,\beta-1\right\}$ con
			      $a_{1}\neq 0$ e $i\in\left\{1,\ldots,t\right\}$.

			\item

			      El exponente $e\in\left\{L,\ldots,U\right\}$ toma $U-L+1$ valores.
		\end{itemize}

		Por el principio de la multiplicación, la cantidad total de
		términos son
		\begin{equation*}
			\operatorname{card}\left(\mathbb{F}\right)=
			2\times
			\left(\beta-1\right)\times
			\beta^{t-1}\times
			\left(U-L+1\right)+
			\operatorname{card}\left(\left\{0\right\}\right)=
			2\left(\beta-1\right)\beta^{t-1}\left(U-L+1\right)+
			1.
		\end{equation*}

		Por otro lado, las cotas inferior y superior del valor absoluto
		de $x$ son
		\begin{align*}
			x_{\min} & =
			{\left(-1\right)}^{0}\cdot
			\left(0.1\ldots 0\right)\cdot
			{\beta}^{L}=
			1\cdot
			{\beta}^{-1}\cdot
			{\beta}^{L}=
			{\beta}^{-1+L}=
			\beta^{L-1}\leq
			\left|x\right|. \\
			x_{\max} & =
			{\left(-1\right)}^{0}\cdot
			\left(0.\beta-1\ldots \beta-1\right)\cdot
			{\beta}^{U}=
			1\cdot
			\sum_{i=1}^{t}
			\left(\beta-1\right)
			\beta^{-i}\cdot
			{\beta}^{U}=
			\left(\beta-1\right)
			{\beta}^{U}
			\sum_{i=1}^{t}
			\beta^{-i}.     \\
			x_{\max} & =
			\left(\beta-1\right)
			{\beta}^{U}
			\left(\frac{1-\beta^{-t}}{\beta-1}\right)
			=\beta^{U}\left(1-\beta^{-t}\right)\geq
			\left|x\right|.
		\end{align*}
	\end{solution}
\end{frame}

\begin{frame}
	\begin{solution}
		Consideremos el conjunto de punto flotante normalizado
		$\mathbb{F}\left(10,3,-2,3\right)$, aplicando la fórmula deducida
		obtenemos que el
		\begin{equation*}
			\operatorname{card}
			\left(
			\mathbb{F}
			\left(\beta=10,t=3,L=-2,U=3\right)
			\right)=2\times
			\left(10-1\right)\times
			10^{3-1}\times
			\left(3-\left(-2\right)+1\right)+
			1=
			2\times 9\times 10^{2}\times 6 + 1
			=10801.
		\end{equation*}

		También,
		\begin{align*}
			x_{\min}     & =
			\beta^{L-1}=
			10^{-2-1}=
			10^{-3}.         \\
			x_{\max}     & =
			\beta^{U}\left(1-\beta^{-t}\right)=
			10^{3}\left(1-10^{-3}\right)=
			999.             \\
			\epsilon_{M} & =
			\beta^{1-U}=
			10^{1-3}=10^{-2}.
		\end{align*}
	\end{solution}
\end{frame}

\begin{frame}\transblindsvertical
	\frametitle{Referencias}
	%------------------------------------------------------------ 1
	\only<1>{
		\begin{itemize}
			\item Libros
			      \nocite{*}
			      \printbibliography[heading=none,keyword=book]
		\end{itemize}
	}
	%------------------------------------------------------------ 2
	\only<2>{
		\begin{itemize}
			\item Artículos científicos
			      \printbibliography[heading=none,keyword=paper]
		\end{itemize}
	}
	%------------------------------------------------------------ 3
	\only<3>{
		\begin{itemize}
			\item Sitios web
			      \printbibliography[heading=none,keyword=online]
		\end{itemize}
	}
\end{frame}

\end{document}