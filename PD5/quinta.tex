%! Aldo Luna, Alexandra Mazzetti, Carlos Aznarán, Edward Canales
%! Universidad Nacional de Ingeniería
%! Facultad de Ciencias
%! Lima, Perú
%! Uso:
%! $ sudo pacman -Syu texlive-most zathura # instalar las dependencias y un visor
%! $ arara quinta
%! $ zathura quinta.pdf
%! Ver https://wiki.archlinux.org/title/TeX_Live
% arara: clean: {
% arara: --> extensions:
% arara: --> ['aux','log','nav',
% arara: --> 'out','snm','toc','pdf']
% arara: --> }
% arara: lualatex: {
% arara: --> interaction: batchmode
% arara: --> }
% arara: lualatex: {
% arara: --> interaction: batchmode
% arara: --> }
% arara: clean: {
% arara: --> extensions:
% arara: --> ['aux','log','nav',
% arara: --> 'out','snm','toc']
% arara: --> }
\PassOptionsToPackage{svgnames}{xcolor}
\documentclass[
  spanish,
  8pt,
  utf8,
  xcolor=table,
  handout,
  aspectratio=169,
  professionalfonts,
  notheorems,
  mathserif,
  % t
]{beamer}
\setbeamersize{text margin left=0pt,text margin right=0pt}
\usepackage[spanish,es-sloppy]{babel}
\spanishdatedel\decimalpoint
\usepackage{mathtools}
\usepackage{nicematrix}
\usepackage{systeme}
\usepackage{enumerate}
\usepackage{multicol}
\usepackage{array}
\usepackage[linesnumbered,ruled,boxed,vlined,spanish]{algorithm2e}
\usepackage{algorithmicx}

\newcolumntype{x}[1]{>{\centering\arraybackslash\hspace{0pt}}p{#1}}

\newcounter{savedenum}
\newcommand*{\saveenum}{\setcounter{savedenum}{\theenumi}}
\newcommand*{\resume}{\setcounter{enumi}{\thesavedenum}}

\setbeamertemplate{navigation symbols}{}
\setbeamertemplate{footline}{}
\setbeamertemplate{headline}{}

\begin{document}

\begin{frame}
	\begin{enumerate}
		\item

		      Encuentre los polinomios de mínimos cuadrados de grados
		      $1$, $2$ y $3$ para los datos en la siguiente tabla.

		      \begin{multicols}{2}
			      \begin{enumerate}[a)]

				      \item

				            \begin{table}[ht!]
					            \centering
					            \begin{tabular}{ccccccc}
						            $x_{i}$ & 0   & 0,15  & 0.31   & 0.5   & 0.6   & 0.75  \\
						            \hline
						            $y_{i}$ & 1.0 & 1.004 & 1.0031 & 1.117 & 1.223 & 1.422
					            \end{tabular}
				            \end{table}

				      \item

				            \begin{table}[ht!]
					            \centering
					            \begin{tabular}{ccccccc}
						            $x_{i}$ & 1.0  & 1.1  & 1.3  & 1.5  & 1.9  & 2.1  \\
						            \hline
						            $y_{i}$ & 1.84 & 1.96 & 2.21 & 2.45 & 2.94 & 3.18
					            \end{tabular}
				            \end{table}
			      \end{enumerate}
		      \end{multicols}
		      Calcule el error $E$ en cada caso.
		      Grafique los datos y los polinomios.
		      Implemente un algoritmo.

		\item

		      Encuentre la aproximación polinomial de mínimos cuadrados
		      lineales en el intervalo $\left[-1,1\right]$ para las
		      siguientes funciones.

		      \begin{multicols}{2}

			      \begin{enumerate}[a)]

				      \item

				            \begin{math}
					            f\left(x\right)=
					            x^{2}-2x+3
				            \end{math}.

				      \item

				            \begin{math}
					            f\left(x\right)=
					            \dfrac{1}{2}\cos\left(x\right)+
					            \dfrac{1}{3}\sin\left(2x\right)
				            \end{math}.

			      \end{enumerate}
		      \end{multicols}
		      Implemente un algoritmo.

		\item

		      Determine el número de un polígono convexo de $n$ lados,
		      con valor inicial en el origen, según el siguiente
		      requerimiento.

		      \begin{multicols}{2}
			      \begin{enumerate}[a)]
				      \item

				            Modele el sistema a resolver.

				      \item

				            Resuelve con el método de Gram-Schmidt y
				            factorización QR.
			      \end{enumerate}
		      \end{multicols}

		\item

		      Una familia consta de una madre, un padre y una hija.
		      La suma de las edades actuales de los $3$ es de $80$ años.
		      Dentro de $22$ años, la edad del hijo será la mitad que la
		      de la madre.
		      Si el padre es un año mayor que la madre.
		      Determinar la edad de la familia según el siguiente
		      requerimiento.

		      \begin{multicols}{2}
			      \begin{enumerate}[a)]
				      \item

				            Modele el sistema a resolver.

				      \item

				            Resuelve con el método de Gram-Schmidt y
				            factorización QR.
			      \end{enumerate}
		      \end{multicols}

		\item

		      En una heladería, por un helado, dos zumos y $4$ batidos
		      nos cobraron $35$ soles.
		      Otro día, por $4$ helados, $4$ zumos y un batido nos
		      cobraron $34$ soles.
		      Un tercer día por $2$ helados, $3$ zumos y $4$ batidos $42$
		      soles.
		      Determine el precio de cada uno según el siguiente
		      requerimiento.

		      \begin{multicols}{2}
			      \begin{enumerate}[a)]
				      \item

				            Modele el sistema a resolver.

				      \item

				            Resuelve con el método de Gram-Schmidt y
				            factorización QR.
			      \end{enumerate}
		      \end{multicols}
		      \saveenum
	\end{enumerate}
\end{frame}

\begin{frame}
	\begin{enumerate}
		\resume

		\item

		      Hay tres tipos de cereal, de los cuales tres fardos del
		      primero, dos del segundo, y uno del tercero hacen $39$
		      medidas.
		      Dos del primero, tres del segundo y uno del tercero hacen
		      $34$ medidas.
		      Y uno del primero, dos del segundo y tres del tercero hacen
		      $26$ medidas.
		      Determine la medida de cereal ue contiene un fardo de cada
		      tipo, según el siguiente requerimiento.


		      \begin{multicols}{2}
			      \begin{enumerate}[a)]
				      \item

				            Modele el sistema a resolver.

				      \item

				            Resuelve con la transformación de Givens.
			      \end{enumerate}
		      \end{multicols}

		\item

		      Un negociante internacional necesita en promedio,
		      cantidades fijas de yenes, francos y marcos para cada uno
		      de sus viajes de negocio.
		      Este año viajó tres veces.
		      La primera vez cambió un total de $\$434$ a la siguiente
		      paridad: $100$ yenes, $1.5$ francos y $1.2$ marcos por
		      dólar.
		      La segunda vez, cambió un total de $\$ 406$ con las
		      siguentes tasas: $100$ yenes, $1.2$ francos y $1.5$ marcos
		      por dólar.
		      La tercera vez cambió $\$ 434$ en total, a 125 yenes, $1.2$
		      francos y $1.2$ marcos por dólar.
		      Determine la cantidad de yenes, francos y marcos que
		      compró, según el siguiente requerimiento.


		      \begin{multicols}{2}
			      \begin{enumerate}[a)]
				      \item

				            Modele el sistema a resolver.

				      \item

				            Resuelve con la transformación de Givens.
			      \end{enumerate}
		      \end{multicols}

		\item

		      Tres industrias interrelacionadas $I_{1}$, $I_{2}$ y
		      $I_{3}$ que producen un único bien cada una y cuya
		      producción se obtiene de la forma siguiente:
		      Cada unidad de $I_{1}$ requiere $0.3$ unidades de $I_{1}$,
		      $0.2$ unidades de $I_{2}$ y $0.3$ unidades de $I_{3}$.
		      Cada unidad producida de $I_{2}$ necesita $0.1$ unidades de
		      $I_{1}$, $0.2$ de $I_{2}$ y $0.3$ de $I_{3}$, y cada unidad
		      de $I_{3}$ precisa $0.2$, $0.5$ y $0.1$ unidades producidas
		      en $I_{1}$, $I_{2}$ e $I_{3}$ respectivamente.
		      Si las demandas exteriores son $45$, $50$ y $51$ unidades
		      de $I_{1}$, $I_{2}$ e $I_{3}$, determine los niveles de
		      producción que permiten el equilibrio de esta economía,
		      según el siguiente requerimiento.

		      \begin{multicols}{2}
			      \begin{enumerate}[a)]
				      \item

				            Modele el sistema a resolver.

				      \item

				            Resuelve con la transformación de Givens.
			      \end{enumerate}
		      \end{multicols}

		\item

		      Demuestre que la transformación Householder, $P=I-2ww^{T}$,
		      es simétrica y ortogonal $\left(P^{T}P=I\right)$, y así,
		      $P^{-1}=P$.
	\end{enumerate}
	\saveenum
\end{frame}

\begin{frame}
	\begin{enumerate}
		\resume

		\item

		      Encuentre por el método de Householder, para encontrar la
		      descomposición $QR$ de la siguiente matriz
		      \begin{math}
			      A=
			      \begin{bNiceMatrix}
				      2  & -1 & -1 & 0  & 0 \\
				      -1 & 3  & 0  & -2 & 0 \\
				      -1 & 0  & 4  & 2  & 1 \\
				      0  & -2 & 2  & 8  & 3 \\
				      0  & 0  & 1  & 3  & 9
			      \end{bNiceMatrix}
		      \end{math}
		      y determine la determinante de la matriz $A$.

		\item

		      Encuentre por el método de Householder, para encontrar la
		      descomposición $QR$ de la siguiente matriz
		      \begin{math}
			      A=
			      \begin{bNiceMatrix}
				      2  & -1 & -1 & 0  & 0 \\
				      -1 & 3  & 0  & -2 & 0 \\
				      -1 & 0  & 4  & 2  & 1 \\
				      0  & -2 & 2  & 8  & 3 \\
				      0  & 0  & 1  & 3  & 9
			      \end{bNiceMatrix}
		      \end{math}
		      y determine la solución de la siguiente ecuación $Ax=b$,
		      donde
		      \begin{math}
			      b=
			      {\begin{bNiceMatrix}
				      -1 \\
				      3  \\
				      1  \\
				      1  \\
				      2
			      \end{bNiceMatrix}}^{T}
		      \end{math}.

		\item

		      Resuelva el sistema $Ax=b$ calculando la factorización QR
		      de la matriz $A$, usando la reducción de Householder:
		      \begin{math}
			      A=
			      \begin{bNiceMatrix}
				      1 & 2 & 2  \\
				      7 & 6 & 10 \\
				      4 & 4 & 6  \\
				      1 & 0 & 1
			      \end{bNiceMatrix},
			      b=
			      \begin{bNiceMatrix}
				      6 \\
				      6 \\
				      8 \\
				      3
			      \end{bNiceMatrix}
		      \end{math}.

		\item

		      Use ahora el método de las rotaciones de Givens para hallar
		      una solución de mínimos cuadrados del problema anterior.

		\item

		      Resuelva el siguiente sistema $Ax=b$ usando factorización
		      QR
		      \begin{math}
			      A=
			      \begin{bNiceMatrix}
				      0  & 2  \\
				      0  & 0  \\
				      -1 & -2
			      \end{bNiceMatrix},
			      b=
			      \begin{bNiceMatrix}
				      1 \\
				      1 \\
				      0
			      \end{bNiceMatrix}
		      \end{math}.
		      Si añadimos la condición de que $R\left(i;i\right)>0$,
		      $1\leq i\leq n$, ¿es cierto que la factorización QR es
		      única?

		\item

		      Calcular ${\left\|A\right\|}_{1}$, ${\left\|A\right\|}_{2}$
		      y ${\left\|A\right\|}_{\infty}$ para las matrices
		      \begin{math}
			      A=
			      \begin{bNiceMatrix}
				      1  & -2 \\
				      -2 & 4
			      \end{bNiceMatrix},
			      A=
			      \begin{bNiceMatrix}
				      4  & 1 & -1 \\
				      1  & 2 & 0  \\
				      -1 & 0 & 2
			      \end{bNiceMatrix}
		      \end{math}.
		      \saveenum
	\end{enumerate}
\end{frame}

\begin{frame}
	\begin{enumerate}
		\resume

		\item

		      Encontrar una raíz de la ecuación
		      \begin{math}
			      f\left(x\right)=
			      x^{3}-x-1
		      \end{math},
		      implementando el método de la bisección.
		      Testee con $0$, $1$ y $2$.

		\item

		      Dada la matriz
		      \begin{math}
			      A=
			      \begin{bNiceMatrix}
				      2 & 1 & 0 \\
				      1 & 2 & 1 \\
				      0 & 1 & 6
			      \end{bNiceMatrix}
		      \end{math}.

		      Realice cuatro iteraciones del método de bisección a mano
		      para estimar el valor propio entre $2$ y $3$.
		      Verifique, implementando el método de la bisección.

		\item

		      Localice los intervalos que contienen las raíces reales
		      positivas de la ecuación $x^{3}-3x+1=0$.
		      Obtenga estas raíces hasta tres decimales correctas, usando
		      el método de la regla falsa.

		\item

		      Encuentra la raíz correcta a dos decimales de la ecuación
		      \begin{math}
			      x\exp\left(x\right)=
			      \cos\left(x\right)
		      \end{math},
		      usando el método de la regla falsa.

		\item

		      Resuelva $2x^{3}-2,5x-5=0$ para la raíz en el intervalo
		      $\left[1,2\right]$ por el método de la regla falsa
		      modificada.

		\item

		      Resuelva $5\sin x^{2}-8\cos ^{5}x=0$ para la raíz en el
		      intervalo $\left[0.5,1.5\right]$ por el método de la regla
		      falsa modificada.
		      \saveenum
	\end{enumerate}
\end{frame}

\end{document}