\begin{frame}
	\frametitle{Formulación matemática del método del gradiente conjugado}
	El \alert{método del gradiente conjugado} puede ser descrito como
	sigue:

	Dado un $x^{\left(0\right)}$,
	$r^{\left(0\right)}=b-Ax^{\left(0\right)}$,
	$p^{\left(0\right)}=r^{\left(0\right)}$ y $\forall k\geq0$ hasta
	convergencia calcule
	\begin{align*}
		\alpha_{k}
		 & =
		\frac{
			{p^{\left(k\right)}}^{T}
			r^{\left(k\right)}
		}{
			{p^{\left(k\right)}}^{T}
		Ap^{\left(k\right)}}, \\
		x^{\left(k+1\right)}
		 & =
		x^{\left(k\right)}+
		\alpha_{k}
		p^{\left(k\right)},   \\
		r^{\left(k+1\right)}
		 & =
		r^{\left(k\right)}-
		\alpha_{k}
		Ap^{\left(k\right)},  \\
		\beta_{k}
		 & =
		\frac{
			{\left(Ap^{\left(k\right)}\right)}^{T}
			r^{\left(k+1\right)}
		}{
			{\left(Ap^{\left(k\right)}\right)}^{T}
		p^{\left(k\right)}},  \\
		p^{\left(k+1\right)}
		 & =
		r^{\left(k+1\right)}-
		\beta_{k}p^{\left(k\right)}.
	\end{align*}

	\begin{theorem}[Convergencia del método del gradiente conjugado]
		Sea $A$ una matriz simétrica y definida positiva.
		El método del gradiente conjugado para resolver $Ax=b$ converge
		después de un máximo de $n$ pasos.
	\end{theorem}
\end{frame}

\begin{frame}
	\begin{enumerate}\setcounter{enumi}{17}
		\item

		      Un fabricante de autos ha lanzado al mercado tres nuevos
		      modelos $A, B$ y $C$.
		      El precio de venta de cada modelo es de $15000$, $20000$ y
		      $30000$ soles, respectivamente, ascendiendo el importe
		      total de los autos vendidos durante el primer mes a
		      $250000$.
		      Donde los costes de fabricación son de S/$10000$ para el
		      modelo $A$, S/$15000$ para el modelo $B$ y S/$20000$ para
		      el modelo $C$, siendo el coste total de fabricación de los
		      autos en el mes de S/$175000$ y el número total de autos
		      vendidos es de $140$.
		      Determine el número de autos vendidos de cada modelo en el
		      mes, usando el programa desarrollado del gradiente conjugado.

	\end{enumerate}
	\begin{solution}

		\

		\begin{minipage}{0.55\textwidth}
			\begin{itemize}
				\item

				      $x$: representa el número de autos vendidos del modelo $A$.

				\item

				      $y$: representa el número de autos vendidos del modelo $B$.

				\item

				      $z$: representa el número de autos vendidos del modelo $C$.
			\end{itemize}
		\end{minipage}
		\begin{minipage}{0.35\textwidth}
			\systeme{
				15000x+20000y+30000z=250000,
				10000x+15000y+20000z=175000,
				x+y+z=140
			}
		\end{minipage}

		\

		Sea
		\begin{math}
			A=
			\begin{bNiceMatrix}
				15000 & 20000 & 30000 \\
				10000 & 15000 & 20000 \\
				1     & 1     & 1
			\end{bNiceMatrix}
		\end{math}
		pero \alert{no es simétrica} y \alert{no es definida positiva}.

		Por lo que vamos a multplicar por $A^{T}$ al sistema $Ax=b$, resultando
		\begin{align*}
			Ax             & =b             \\
			A^{T}Ax        & =A^{T}b        \\
			\widetilde{A}x & =\widetilde{b}
		\end{align*}

	\end{solution}
\end{frame}

\begin{frame}
	\begin{solution}
		\begin{minipage}{0.45\textwidth}
			\begin{listing}[H]
				\inputminted[
					fontsize=\tiny,
					breaklines,
					firstline=1,
					lastline=10
				]{text}{resultado_pregunta18.txt}
			\end{listing}
		\end{minipage}
		\begin{minipage}{0.45\textwidth}
			\begin{listing}[H]
				\inputminted[
					fontsize=\scriptsize,
					breaklines,
					firstline=1,
					lastline=30
				]{python}{edward_gradiente_conjugado.py}
			\end{listing}
		\end{minipage}
	\end{solution}
\end{frame}