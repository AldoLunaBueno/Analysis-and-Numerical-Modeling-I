\begin{frame}
	\begin{enumerate}\setcounter{enumi}{17}
		\item

		      Un fabricante de autos ha lanzado al mercado tres nuevos
		      modelos $A, B$ y $C$.
		      El precio de venta de cada modelo es de $15000$, $20000$ y
		      $30000$ soles, respectivamente, ascendiendo el importe
		      total de los autos vendidos durante el primer mes a
		      $250000$.
		      Donde los costes de fabricación son de S/$10000$ para el
		      modelo $A$, S/$15000$ para el modelo $B$ y S/$20000$ para
		      el modelo $C$, siendo el coste total de fabricación de los
		      autos en el mes de S/$175000$ y el número total de autos
		      vendidos es de $140$.
		      Determine el número de autos vendidos de cada modelo en el
		      mes, usando el programa desarrollado del gradiente conjugado.

	\end{enumerate}
	\begin{solution}

		\

		\begin{minipage}{0.55\textwidth}
			\begin{itemize}
				\item

				      $x$: representa el número de autos vendidos del modelo $A$.

				\item

				      $y$: representa el número de autos vendidos del modelo $B$.

				\item

				      $z$: representa el número de autos vendidos del modelo $C$.
			\end{itemize}
		\end{minipage}
		\begin{minipage}{0.35\textwidth}
			\systeme{
				15000x+20000y+30000z=250000,
				10000x+15000y+20000z=175000,
				x+y+z=140
			}
		\end{minipage}
	\end{solution}
\end{frame}

\begin{frame}
	\begin{solution}
		\begin{minipage}{0.45\textwidth}
			\begin{listing}[H]
				\inputminted[
					fontsize=\tiny,
					breaklines,
					firstline=1,
					lastline=10
				]{text}{resultado_pregunta18.txt}
			\end{listing}
		\end{minipage}
		\begin{minipage}{0.45\textwidth}
			\begin{listing}[H]
				\inputminted[
					fontsize=\scriptsize,
					breaklines,
					firstline=1,
					lastline=30
				]{python}{edwin_gradiente_conjugado.py}
			\end{listing}
		\end{minipage}
	\end{solution}
\end{frame}