%! Aldo Luna, Alexandra Mazzetti, Carlos Aznarán, Edward Canales
%! Universidad Nacional de Ingeniería
%! Facultad de Ciencias
%! Lima, Perú
%! Uso:
%! $ sudo pacman -Syu texlive-most zathura # instalar las dependencias y un visor
%! $ arara cuarta
%! $ zathura cuarta.pdf
%! Ver https://wiki.archlinux.org/title/TeX_Live
% arara: clean: {
% arara: --> extensions:
% arara: --> ['aux','log','nav',
% arara: --> 'out','snm','toc','pdf']
% arara: --> }
% arara: lualatex: {
% arara: --> interaction: batchmode
% arara: --> }
% arara: lualatex: {
% arara: --> interaction: batchmode
% arara: --> }
% arara: clean: {
% arara: --> extensions:
% arara: --> ['aux','log','nav',
% arara: --> 'out','snm','toc']
% arara: --> }
\PassOptionsToPackage{svgnames}{xcolor}
\documentclass[
	spanish,
	8pt,
	utf8,
	xcolor=table,
	handout,
	aspectratio=169,
	professionalfonts,
	notheorems,
	mathserif,
	% t
]{beamer}
\setbeamersize{text margin left=0pt,text margin right=0pt}
\usepackage[spanish,es-sloppy]{babel}
\spanishdatedel\decimalpoint
\usepackage{mathtools}
\usepackage{nicematrix}
\usepackage{systeme}
\usepackage{enumerate}
\usepackage{multicol}
\usepackage{array}
\usepackage[linesnumbered,ruled,boxed,vlined,spanish]{algorithm2e}
\usepackage{algorithmicx}

\newcolumntype{x}[1]{>{\centering\arraybackslash\hspace{0pt}}p{#1}}

\newcounter{savedenum}
\newcommand*{\saveenum}{\setcounter{savedenum}{\theenumi}}
\newcommand*{\resume}{\setcounter{enumi}{\thesavedenum}}

\setbeamertemplate{navigation symbols}{}
\setbeamertemplate{footline}{}
\setbeamertemplate{headline}{}

\begin{document}

\begin{frame}
	\begin{enumerate}
		\item

		      Un fabricante de bombillas gana $0.3$ dólares por cada
		      bombilla que sale de la fábrica, pero pierde $0.4$ dólare
		      por cada una que sale defectuosa.
		      Un día en el que fabricó 2100 bombillas obtuvo un beneficio
		      de $484.4$ dólares.
		      Determine el número de bombillas buenas y defectuosa según
		      el siguiente requerimiento.

		      \begin{multicols}{2}
			      \begin{enumerate}[a)]

				      \item

				            Modele el problema.


				      \item

				            Resolver usando el método de Richardson y Jacobi.

			      \end{enumerate}
		      \end{multicols}

		\item

		      Sean dos números tales que la suma de un tercio del primero
		      más un quinto del segundo sea igual a $13$ y que si se
		      multiplica el primero por $5$ y el segundo por $7$ se
		      obtiene $247$ como suma de los dos productos.
		      Determine los números según el requerimiento siguiente.

		      \begin{enumerate}[a)]

			      \item

			            Modele el problema.

			      \item

			            Resolver usando el método de Richardson y Jacobi.
		      \end{enumerate}

		\item

		      Dos kilos de plátanos y tres de peras cuestan $8.80$ soles.
		      Cinco kilos de plátanos y cuatro de peras cuestan $16.40$
		      soles.
		      Determine el costo de kilo del plátano y de la pera según
		      el siguiente requerimiento.

		      \begin{multicols}{2}
			      \begin{enumerate}[a)]
				      \item

				            Modele el problema.

				      \item

				            Resolver usando el método de Richardson y Jacobi.
			      \end{enumerate}
		      \end{multicols}

		\item

		      La edad de Manuel es el doble de la edad de su hija Ana.
		      Hace diez años, la suma de las edades de ambos era igual a
		      la edad actual de Manuel.
		      Determine la edad de ambos según el requerimiento siguiente.


		      \begin{multicols}{2}
			      \begin{enumerate}[a)]
				      \item

				            Modele el problema.

				      \item

				            Resolver usando el método de Richardson y Jacobi.
			      \end{enumerate}
		      \end{multicols}
		      \saveenum
	\end{enumerate}
\end{frame}

\begin{frame}
	\begin{enumerate}
		\resume

		\item

		      José dice a Eva:
		      Mi colección de discos compactos es mejor que la tuya ya
		      que si te cedo 10 tendríamos la misma cantidad.
		      Eva le responde:
		      Reconozco que tienes razón.
		      Solo te faltan 10 para doblarme en número.
		      Determine la cantidad de discos que tiene cada uno según el
		      requerimiento siguiente.

		      \begin{multicols}{2}
			      \begin{enumerate}[a)]
				      \item

				            Modele el problema.

				      \item

				            Resolver usando el método de Richardson y Jacobi.
			      \end{enumerate}
		      \end{multicols}

		\item

		      Un cliente de un supermercado ha pagado un total de S/$156$
		      por 24 litros de leche, 6 kilogramos de jamón serrano y 12
		      litros de aceite de oliva.
		      Determine el precio de cada artículo, sabiendo que 1 litro
		      de aceite cuesta el triple que 1 litro de leche y que 1
		      kilogramo de jamón cuesta igual que 4 litros de aceite más
		      4 litros de leche.
		      Usando los programas desarrollados de Jacobi y Gauss-Seidel.

		\item

		      Un video club está especializado en tres tipos de
		      películas: Infantiles, Oeste y Terror.
		      Donde el $60$\% de las películas Infantiles más el
		      $50$\% de las del Oeste representan el $30$\% del total de
		      las películas.
		      Además el $20$\% de las Infantiles más el $60$\% de las del
		      Oeste más del $60$\% de las de Terror representan la mitad
		      del total de las películas.
		      Hay 100 películas más del Oeste que de Infantiles.
		      Determine el número de películas de cada tipo usando los
		      programas desarrollados de Jacobi y Gauss-Seidel.

		\item

		      Una familia consta de una madre, un padre y un hijo.
		      La suma de las edades actuales de los tres es de $80$ años.
		      Dentro de 22 años, la edad del hijo será la mitad que la
		      madre.
		      Si el padre es un año mayor que la madre.
		      Determine la edad actual de cada miembro de la familia
		      usando los programas desarrollados de Jacobi y
		      Gauss-Seidel, con
		      \begin{math}
			      \overline{x}_{0}=
			      {\left(1,2,0\right)}^{T}
		      \end{math}.

		\item

		      Una empresa de transportes gestiona una flota de $60$
		      camiones de tres modelos diferentes.
		      Los mayores transportan una media diaria de $15000$
		      kilogramos y recorren diariamente una media de $400$
		      kilómetros.
		      Los medianos transportan diariamente una media de $10000$
		      kilogramos y recorren $300$ kilómetros.
		      Los pequeños transportan diariamente $5000$ kilogramos y
		      recorren $100$ kilómetros de media.
		      Diariamente los camiones de la empresa transportan un total
		      de $475$ toneladas y recorren $12500$ kilómetros entre todos.
		      Determine el número de camiones que gestiona la empresa de
		      cada modelo, usando los programas desarrollados de Jacobi y
		      Gauss-Seidel.
		      \saveenum
	\end{enumerate}
\end{frame}

\begin{frame}
	\begin{enumerate}
		\resume
		\item

		      Un carpintero fabrica sillas, mesas redondas y mesas
		      cuadradas.
		      Cada silla requiere un minuto de lija, 3 minutos de pintura
		      y un minuto de barniz.
		      Cada mesa redonda requiere 2 minutos de lija, 1 minuto de
		      pintura y un minuto de barniz.
		      Cada mesa cuadrada requiere un minuto de lija, 1 minuto de
		      pintura y 3 minutos de barniz.
		      Las máquinas de lijar, pintar y barnizar están disponibles
		      $6$, $6$ y $5$ horas por día respectivamente.
		      Determine el número de muebles que puede fabricar el
		      carpintero si las máquinas se usan a toda capacidad, usando
		      los programas desarrollados de Jacobi y Gauss-Seidel.

		\item

		      Resuelva el siguiente sistema lineal
		      \begin{math}
			      \systeme{
			      4x_{1}-
			      x_{2}-
			      x_{4}=
			      0,
			      -x_{1}+
			      4x_{2}-
			      x_{3}-
			      x_{5}=
			      5,
			      -x_{2}+
			      4x_{3}-
			      x_{6}=
			      0,
			      -x_{1}+
			      4x_{4}-
			      x_{5}=
			      6,
			      -x_{2}-
			      x_{4}+
			      4x_{5}-
			      x_{6}=
			      -2,
			      -x_{3}-
			      x_{5}+
			      4x_{6}=
			      6
			      }
		      \end{math}
		      tiene solución ${\left(1,-1,1\right)}^{T}$.

		      Resuelva el sistema lineal mediante los métodos de SOR y
		      del descenso más rápido con una aritmética de redondeo a
		      tres dígitos.
		      \saveenum
	\end{enumerate}
\end{frame}

\begin{frame}
	\begin{enumerate}
		\resume

		\item

		      Use los métodos de SOR y del descenso más rápido para
		      encontrar la solución de sistema $Ax=b$ con una precisión
		      de $10^{-5}$ en la norma ${\left\|\cdot\right\|}_{\infty}$.
		      % \setcounter{MaxMatrixCols}{30}
		      \begin{equation*}
			      a_{ij}=
			      \begin{cases}
				      4,  & \text {si }j=1\text { e }i=1,\ldots,16,      \\
				      -1, & \text {si }
				      \begin{cases}
					      j=i+1 \text { e } i=1,2,3,5,6,7,9,10,11,13,14,15,  \\
					      j=i-1 \text { e } i=2,3,4,6,7,8,10,11,12,14,15,16, \\
					      j=i+4 \text { e } i=1,\ldots, 12,                  \\
					      j=i-4 \text { e } i=5,\ldots, 16,
				      \end{cases} \\
				      0,  & \text {en otro caso}.
			      \end{cases}\quad
			      \text{y}\quad
			      b=
			      \begin{bNiceMatrix}
				      1.902207  \\
				      1.051143  \\
				      1.175689  \\
				      3.480083  \\
				      0.819600  \\
				      -0.264419 \\
				      -0.412789 \\
				      1.175689  \\
				      0.913337  \\
				      -0.150209 \\
				      -0.264419 \\
				      1.051143  \\
				      1.966694  \\
				      0.913337  \\
				      0.819600  \\
				      1.902207
			      \end{bNiceMatrix}
		      \end{equation*}
		      \saveenum
	\end{enumerate}
\end{frame}

\begin{frame}
	\begin{enumerate}
		\resume

		\item

		      Use los métodos de SOR y del descenso más rápido para
		      encontrar la solución de sistema $Ax=b$ con una precisión
		      de $10^{-5}$ en la norma ${\left\|\cdot\right\|}_{\infty}$.

		      \begin{equation*}
			      a_{ij}=
			      \begin{cases}
				      4,  & \text {si }j=1\text { e }i=1,\ldots,25,                             \\
				      -1, & \text {si }
				      \begin{cases}
					      j=i+1 \text { e } i=1,2,3,4,6,7,8,9,11,12,13,14,16,17,18,19,21,22,23,24,  \\
					      j=i-1 \text { e } i=2,3,4,5,7,8,9,10,12,13,14,15,17,18,19,20,22,23,24,25, \\
					      j=i+5 \text { e } i=1,\ldots, 20,                                         \\
					      j=i-5 \text { e } i=6,\ldots, 25,
				      \end{cases} \\
				      0,  & \text {en otro caso}.
			      \end{cases}\quad
			      \text{y}\quad
			      b=
			      \begin{bNiceMatrix}
				      1  \\
				      0  \\
				      -1 \\
				      0  \\
				      2  \\
				      1  \\
				      0  \\
				      -1 \\
				      0  \\
				      2  \\
				      1  \\
				      0  \\
				      -1 \\
				      0  \\
				      2  \\
				      1  \\
				      0  \\
				      -1 \\
				      0  \\
				      2  \\
				      1  \\
				      0  \\
				      -1 \\
				      0  \\
				      2  \\
			      \end{bNiceMatrix}
		      \end{equation*}

		\item

		      Use los métodos de SOR y del descenso más rápido para
		      encontrar la solución de sistema $Ax=b$ con una precisión
		      de $10^{-5}$ es la norma ${\left\|\cdot\right\|}_{\infty}$.


		      \begin{equation*}
			      a_{ij}=
			      \begin{cases}
				      2i, & \text {si }j=1\text { e }i=1,\ldots,40, \\
				      -1, & \text {si }
				      \begin{cases}
					      j=i+1 \text { e } i=1,\ldots,39 \\
					      j=i-1 \text { e } i=2,\ldots,40
				      \end{cases}               \\
				      0,  & \text {en otro caso}.
			      \end{cases}\quad
			      \text{y}\quad
			      b_{i}=
			      \begin{bNiceMatrix}
				      1.5i-6
			      \end{bNiceMatrix}
			      \forall i\in\left\{1,\ldots,40\right\}.
		      \end{equation*}

		\item

		      Demuestre que un conjunto $A$ ortogonal de vectores no
		      nulos asociados con una matriz definida positiva es
		      linealmente independiente.

		\item


		      Una heladería, reabrirá sus puertas luego de $6$ meses de
		      confinamiento debido al covid-19; para ello, realizará una
		      campaña virtual mostrando sus instalaciones en las cuales
		      brinda todas las medidas de seguridad a sus cliente y
		      ofrece los siguientes combos.
		      Combo 1, un helado, 2 zumos y 4 batidos por el módico
		      precio de S/$41.80$.
		      Combo 2 compuesto, por 4 helados, 4 zumos y un batido por
		      un valor de S/$38.00$.
		      Combo 3, 2 helados, 3 zumos y 4 batidos por S/$49.40$.
		      Determine el precio de cada producto usando el programa
		      desarrollado del gradiente conjugado.

		\item

		      Se venden 3 especies de cereales trigo, cebada y mijo.
		      El trigo se vende cada saco por S/$253.50$.
		      La cebada se vende S/$126.75$.
		      El miro se vende cada saco a S/$31.75$.
		      Si se venden 100 sacos y se obtiene por la venta
		      S/$1014$.
		      Determine la cantidad de cada cereal se vendió, usando el
		      programa desarrollado del gradiente conjugado.

		\item


		      Un fabricante de autos ha lanzado al mercado tres nuevos
		      modelos $A, B$ y $C$.
		      El precio de venta de cada modelo es de $15000$, $20000$ y
		      $30000$ soles, respectivamente, ascendiendo el importe
		      total de los autos vendidos durante el primer mes a
		      $250000$.
		      Donde los costes de fabricación son de S/$10000$ para el
		      modelo $A$, S/$15000$ para el modelo $B$ y S/$20000$ para
		      el modelo $C$, siendo el coste total de fabricación de los
		      autos en el mes de S/$175000$ y el número total de autos
		      vendidos es de $140$.
		      Determine el número de autos vendidos de cada modelo en el
		      mes, usando el programa desarrollado del gradiente conjugado.

		\item

		      La ciudad de Huánuco compra $540000$ barriles de petróleo
		      tres suministradores diferentes que lo venden a $27$, $28$
		      y $31$ dólares el barril, respectivamente.
		      La factura total asciende a $15999000$ dólares.
		      Si del primer suministrador recibe el $30$\% del total de
		      petróleo comprado.
		      Determine la cantidad comprada a cada suministrador, usando
		      el programa desarrollado del gradiente conjugado.

		\item

		      Las edades de tres hermanos tales que el quíntuplo de la
		      edad del primero, más el cuádruplo de la edad del segundo,
		      más el triple de la edad del tercero, es igual a $60$.
		      El cuádruplo de la edad del primero, más el triple de la
		      edad del segundo, más el quíntuplo de la edad del tercero,
		      es igual a $50$.
		      Y el triple de la edad del primero, más el quíntuplo de la
		      edad del segundo, más el cuádruplo de la edad del tercero,
		      es igual a $46$.
		      Determine la edad de los tres hermanos usando el programa
		      desarrollado del gradiente conjugado.

		      \saveenum
	\end{enumerate}
\end{frame}

\end{document}